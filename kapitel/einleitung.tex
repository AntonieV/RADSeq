% kapitel1.tex
\chapter{Einleitung} \label{sec:introduction}
\section{Biologischer Hintergrund} \label{sec:biology}
\subsection{Aufbau und Struktur der DNA} \label{subsec:dna}

In den vergangenen Jahren wurden durch die molekulargenetische Methoden in Medizin und Biologie enorme Fortschritte erzielt. Heute sind sie nicht nur ein wesentliches Instrument bei der Erforschung, Diagnostik und Therapie verschiedenster Erkrankungen sondern sind auch bei der Entdeckung und Klassifikation von Organismen oder ganzen Ökosystemen von entscheidender Bedeutung.  \\

%\hyperlink{repl}{DNA-Replikation} 
%\hypertarget{repl}{DNA-Replikation}

Einer der ersten und wichtigsten Meilensteine auf dem noch eher jungen Gebiet der Molekulargenetik wurde $1953$ durch die Entdeckung der Doppelhelixstruktur der DNA und die Beschreibung ihres Aufbaus erreicht ~\cite{watson_1953}. Die DNA (desoxyribonucleic acid) ist ein langkettiges, aus zwei gegenläufigen, komplementären Strängen bestehendes und zu einer Helix gewundenes Molekül, welches die Erbinformation der meisten Zellen codiert. Die Komplementarität und Gegenläufigkeit werden in Kap. \ref{subsec:double_strand} gesondert beschrieben. Jeder Strang besteht aus vielen aneinandergereihten Nukleotiden, die sich jeweils aus einem Zuckermolekül (Desoxyribose), einem Phosphatrest und einer von vier möglichen Basen zusammensetzen. Als Basen kommen in der DNA Adenin (A), Thymin (T), Guanin (G) und Cytosin (C) vor. Die Kombinationen dieser Basen codieren über ihre Sequenz die genetische Information.  \\

Hierbei codieren im Rahmen der Proteinbiosynthese (Kap. \ref{subsec:protsynth}) Kombinationen aus jeweils drei Basen, sog. Basentripletts bzw. Codons, entweder für eine von i.d.R. $ 20 $ Aminosäure \cite{martin_1961, matthaei_1961} oder für Start- bzw. Stop-Sequenzen, welche den Anfang bzw. das Ende von Genen signalisieren (genetischer Code). Hiervon codieren $ 61 $ Tripletts für die bereits erwähnten $ 20 $ Aminosäuren. Somit codieren für die meisten Aminosäuren mehrere verschiedene Tripletts, diese Eigenschaft des genetischen Codes wird auch als Degeneration bezeichnet. \\

Die Basensequenz der DNA entspricht somit einer Aminosäuresequenz, welche die Primärstruktur von Proteinen (Eiweißmolekülen) darstellt. Proteine erfüllen in lebenden Organismen umfangreiche Funktionen, sie können als Hormone, Enzyme, Strukturproteine fungieren und sind an den meisten Stoffwechselprozessen und Signalwegen von Zellen beteiligt. Die verschiedenen Proteine werden jeweils von für sie spezifischen Abschnitten auf der DNA codiert. Solche DNA-Abschnitte werden als Gene bezeichnet. Ein DNA-Strang besteht aus vielen verschiedenen Genen. In komplexeren Zellen befinden sich im Zellkern in der Regel mehrere DNA-Stränge, die jeweils ein Chromosom repräsentieren, auf dem sich jeweils mehrere Gene befinden. Die Anzahl der Chromosomen innerhalb der Zellen ist speziesabhängig. \\

Zwischen den einzelnen Genen eines DNA-Stranges befinden sich nicht-codierende und oft repetitive Sequenzen. Ebenso gibt es auch innerhalb der Gene codierende Abschnitte (Exons) und nicht-codierende Sequenzen (Introns). Insgesamt machen die für Proteine codierenden Bereiche der DNA nur einen geringen Anteil des Genoms, also der gesamten Erbinformation einer Zelle aus. Beim Menschen wird dieser Anteil auf etwa $2$ \% geschätzt, d.h. ca. $ 98 $ \% des menschlichen Genoms besteht aus DNA, die nicht für Proteine codiert. Über diese nicht-codierenden Abschnitte ist bislang nur wenig bekannt, teilweise werden ihnen regulatorische Funktionen zugeschrieben ~\cite{dunham_2012, tsagakis_2020}. \\

\subsection{Bindungen innerhalb und zwischen DNA-Molekülen} \label{subsec:double_strand}
%Besonderheiten doppelsträngiger DNA

DNA liegt meist in Form eines Doppelstrangs vor. Die Basen beider Stränge sind dabei intermolekular über schwache chemische Bindungen, sogenannte Wasserstoffbrücken, mit einander verbunden. Dabei kann Adenin nur an Thymin unter Ausbildung von zwei Wasserstoffbrückenbindungen binden. Ebenso kann Cytosin nur mit Guanin über insgesamt drei Wasserstoffbrücken eine Bindung eingehen. Diese selektive Basenpaarung wird auch als Komplementarität bezeichnet. Es sind also Adenin und Thymin ebenso wie Cytosin und Guanin jeweils komplementär zu einander. Bezogen auf einen DNA-Doppelstrang sind auch seine beiden Einzelstränge komplementär, so das sich für jede Base des einen Stranges auf dem anderen Strang jeweils die komplementäre Base an der entsprechenden Position befindet. Es genügt also die Sequenz von einem der beiden Stränge zu kennen, um die Sequenz des jeweils anderen rekonstruieren zu können. Dies wird sowohl zum Auslesen der genetischen Informationen bei der Transkription (Kap. \ref{subsec:protsynth}) als auch zum Kopieren von DNA im Rahmen der DNA-Replikation (Kap. \ref{subsec:replication}) genutzt. \\

Wie bereits in Kap. \ref{subsec:dna} erwähnt, ist doppelsträngige DNA gegenläufig orientiert. Die Einzelstränge besitzen eine Polarität, welche durch die intramolekulare Bindung zwischen den einzelnen Nukleotiden über sogenannte Phosphodiesterbindungen zustande kommt. Dabei bindet der Phosphatrest, der sich jeweils am 5. Kohlenstoffatom (C5) des Zuckermoleküls der Nukleotide befindet, an das 3. Kohlenstoffatom (C3) des Zuckermoleküls des nachfolgenden Nukleotids. An den Enden eines DNA-Strangs fehlt jedoch diese Phosphodiesterbindung, so das an einem Ende das C3 ungebunden bleibt (3'-Ende) während am anderen Ende der Phosphatrest nur an C5 gebunden ist (5'-Ende) und somit die zweite Esterbindung am Phosphatrest fehlt. Aufgrund der Gegenläufigkeit befindet sich also an beiden Enden eines Doppelstrangs jeweils ein 3'-Ende des einen und ein 5'-Ende des anderen Einzelstrangs. Diese Polarität spielt eine wichtige Rolle bei der Lese- und Synthese-Richtung im Rahmen der Transkription und der DNA-Replikation.\\

\subsection{RNA und die Proteinbiosynthese} \label{subsec:protsynth}

Ebenso wie DNA gehört auch RNA (ribonucleic acid) zu den Nukleinsäuren. Sie unterscheidet sich in ihrem Aufbau von der DNA durch die Base Uracil (U) statt Thymin und den Zucker Ribose statt Desoxyribose. Meist liegt RNA einzelsträngig oder nur über kürzere Abschnitte doppelsträngig vor. Während DNA insbesondere der Speicherung der Erbinformation dient, hat RNA eher die Funktion der Informationsübertragung. RNA nimmt daher zahlreiche regulatorische Funktionen war. \\

Eine ihrer wichtigsten Aufgaben ist die Übertragung der genetischen Information von der DNA in die Aminosäuresequenz der Proteine bei der Proteinbiosynthese. Dabei werden von dem für das herzustellende Protein codierenden DNA-Abschnitt zunächst Arbeitskopien in Form von mRNA (messenger RNA) hergestellt. Ein solches Umschreiben von DNA in mRNA wird auch als Transkription bezeichnet. Nach der Transkription erhalten die mRNA-Fragmente noch einige Modifikationen und werden aus dem Zellkern hinaus in das Cytoplasma transportiert. Im Cytoplasma erfolgt schließlich mit Hilfe sogenannter tRNAs (transfer RNA) die Übersetzung der Basentripletts in eine Aminosäuresequenz (Translation, siehe auch Kap. \ref{subsec:dna}). tRNAs besitzen eine Bindungsstelle bestehend aus jeweils drei Nukleotiden, mit der sie komplementär an ein passendes Basentriplett der mRNA binden. In Abhängigkeit vom Basentriplett an ihrer mRNA-Bindungsstelle trägt jede tRNA entsprechend dem genetischen Code eine spezifischen Aminosäure. Entlang der mRNA wird nun ab der Startsequenz für jedes Basentriplett die passende tRNA nacheinander angelagert. Sobald eine tRNA bindet, wird die an sie gebundene Aminosäure gelöst und an die Aminosäure der nachfolgenden tRNA gebunden. Dadurch entsteht eine Kette von aneinander gebundene Aminosäuren, die sich jeweils entsprechend der mRNA-Sequenz an die Aminosäure der nächsten passenden tRNA anlagert und um deren Aminosäure verlängert. Beim Erreichen einer Stop-Sequenz kann diese keine tRNA angelagern, die Synthese wird abgebrochen und die Aminosäurekette löst sich von der zuletzt gebundenen tRNA und wird weiteren Modifikationen unterzogen. \\

\subsection{DNA-Replikation} \label{subsec:replication}

Die DNA-Replikation dient der Verdopplung der DNA im Rahmen der Zellteilung, so dass jede der beiden resultierenden Tochterzellen das gleiche genetische Material erhält. Die DNA-Replikation ist also ein natürlicher Vorgang zur Erzeugung von DNA-Kopien. Ihr grundlegendes Prinzip findet bei der PCR (Polymerase-Ketten-Reaktion, siehe Kap. \ref{subsec:pcr}) Anwendung und wird für verschiedene molekulargenetische Verfahren genutzt, um DNA-Kopien synthetisch herzustellen. In diesem Zusammenhang sei hier auf die verschiedenen Sequenzierungstechniken insbesondere im Hinblick auf das RAD-Sequencing verwiesen (Kap. \ref{subsec:sanger}, Kap. \ref{subsec:ngs} und Kap. \ref{sec:rad}). Daher soll der Vorgang der DNA-Replikation im Folgenden kurz umrissen werden \cite{odonell_2013, chargin_2010, prioleau_2016}. \\

Bei eukaryotischen Zellen, die im Gegensatz zu Bakterien (Prokaryoten) einen Zellkern besitzen, liegt die DNA im Zellkern häufig gewunden und stark kondensiert vor. Um ihre Sequenz Base für Base kopieren zu können, muss sie zunächst entwunden werden. Dies geschieht durch Enzyme aus der Gruppe der Topoisomerasen. Diese erzeugen gezielt am Replikationsursprung temporäre Strangbrüche in der DNA und entspannen so den Doppelstrang. Anschließend setzt am Replikationsursprung ein weiteres Enzym, die Helikase, an und trennt die beiden komplementären Stränge auf. Es entsteht die sogenannte Replikationsgabel. Während der Replikation schiebt sich die Helikase unter fortlaufender Auftrennung der beiden Stränge auf der DNA entlang. Auch bei diesem Prozess sorgen Topoisomerasen immer wieder für eine Entspannung des DNA-Fadens hinsichtlich seiner Windung. \\

Nun kann der eigentliche Kopiervorgang an den beiden von einander getrennten Einzelsträngen mit Hilfe von DNA-Polymerasen erfolgen. Dabei fahren die DNA-Polymerasen an den Strängen entlang und fügen an jeder Position des Elternstranges Nukleotide mit der jeweils komplementären Base an. Im Ergebnis entsteht also an jedem der beiden Elternstränge ein neuer komplementärer DNA-Strang, der also die gleiche Basensequenz wie der jeweils andere Elternstrang besitzt. Bei den DNA-Polymerasen handelt es sich um Enzyme, die für die Initiation des Kopiervorgangs eine Startsequenz benötigen. Diese Startsequenzen sind kleine RNA-Fragmente mit spezifischer Basensequenz, die sich an die jeweils passende, komplementäre Stelle auf dem Elternstrang anlagern. An diese, auch als RNA-Primer bezeichneten Fragmente kann nun die DNA-Polymerase binden und mit der Replikation beginnen. \\

Die DNA-Polymerase kann den Elternstrang nur in eine Richtung lesen, nämlich vom 3' Ende zum 5' Ende (3'-5'-Richtung). Aufgrund der gegenläufigen, antiparallelen Ausrichtung zweier komplementärer Stränge zu einander, kann folglich die Synthese des Tochter-stranges nur in 5'-3'-Richtung erfolgen. Für die beiden antiparallel ausgerichteten Elternstränge bedeutet dies, dass nur bei einem Strang die Richtung der sich öffnenden Replikationsgabel der 5'-3'-Richtung des Stranges entspricht. Dieser Strang kann kontinuierlich repliziert werden, da sich die DNA-Polymerase auf ihm in Richtung der voranschreitenden Aufspaltung des Doppelstranges bewegt. Der auf diese Weise kontinuierlich synthetisierte Strang wird als Leitstrang bezeichnet. \\

Der andere Strang ist jedoch in Gegenrichtung orientiert, so dass seine Syntheserichtung, also die 5'-3'-Richtung, entgegengesetzt zur Bewegungsrichtung der Replikationsgabel orientiert ist. Dadurch können jeweils nur kleinere Fragmente synthetisiert werden die von der Replikationsgabel bis zum bereits replizierten Teil des Strangs reichen. Schreitet die Öffnung der Replikationsgabel weiter fort, muss der nun frei gewordene Strangabschnitt ebenfalls synthetisiert werden. Es muss also erneut ein RNA-Primer angelagert werden und dann mit Hilfe der DNA-Polymerase der Bereich zwischen Replikationsgabel und bereits repliziertem Strang synthetisiert werden. Die Replikation erfolgt somit diskontinuierlich. Der so synthetisierte Strang wird als Folgestrang bezeichnet und besteht zunächst aus multiplen Fragmenten (Okazaki-Fragmente). Nach dem Replikationsvorgang werden mit Hilfe der DNA-Polymerase die RNA-Primer durch DNA ersetzt. Im Anschluss werden die multiplen Fragmente des Folgestrangs durch Ligasen zu einem kontinuierlichen Strang verbunden. Im Ergebnis sind also nach Abschluss der DNA-Replikation zwei identische Kopien der beiden Elternstränge entstanden.\\

\subsection{Mutationen und SNPs} \label{subsec:mutation}

Veränderungen in der DNA-Sequenz werden als Mutationen bezeichnet. Sie können durch zellinterne Faktoren verursacht werden, wie beispielsweise Fehler beim Kopiervorgang der DNA (DNA-Replikation) während der Zellteilung. Ebenso können sie durch zahlreiche Umwelteinflüsse entstehen. \\

Mutationen können unterschiedlich große Abschnitte der DNA betreffen, von ganzen Chromosomen oder großen Chromosomenabschnitten über Veränderungen von mehreren Basen bis hin zu sogenannten Punktmutationen, bei denen nur eine einzige Base verändert ist. Auf DNA-Ebene können Punktmutationen in Form von Substitutionen, Insertionen und Deletionen auftreten. Bei der Substitution wird eine Base durch eine andere ausgetauscht, bei der Insertion wird eine zusätzliche Base in den DNA-Strang eingefügt und bei der Deletion kommt es zum Verlust einer Base. \\

Liegt eine solche Punktmutation in den codierenden DNA-Abschnitten, so können sich auf Proteinebene verschiedene Konsequenzen daraus ergeben. Punktmutationen in Form von Insertionen und Deletionen (Indels) bewirken durch die zusätzliche bzw. fehlende Base eine Verschiebung des Leserasters, so dass sich die Triplettstruktur für alle nachfolgenden Basen verschiebt. Dies wird als Frame-Shift bezeichnet und verursacht meist eine gravierende Veränderung des resultierenden Proteins, da viele der nachfolgenden Tripletts nun für andere Aminosäuren codieren. Dies führt häufig zu einem deutlich veränderten Protein, welches seine reguläre Funktion nicht mehr oder nur noch unvollständig wahrnehmen kann. Auch bei Indels von mehreren Basen kann es zu einem Frame-Shift kommen, ändert sich dadurch allerdings die Länge des codierenden Abschnitts um drei Basen oder dem Vielfachen hiervon, so bleibt das Leseraster erhalten. \\

Auch bei Substitutionen verändert sich das Leseraster nicht. Aufgrund der Degeneration des genetischen Codes können verschiedene Basentripletts für die gleiche Aminosäure codieren. Dadurch kann ein Basentriplett mit einer Punktmutation trotz des Basenaustauschs noch für die ursprüngliche Aminosäure codieren, so dass das resultierende Protein unverändert bleibt. In diesem Fall spricht man von einer silent-Mutation. Codiert das Basentriplett aber aufgrund der Mutation für eine andere Aminosäure, so handelt es sich um eine missense-Mutation. Die Proteinsequenz wird dadurch in einer Aminosäure geändert, so dass es je nach Position der betreffenden Aminosäure zu verschieden starken Effekten hinsichtlich der Proteinfunktion kommen kann. \\

Zudem können Substitutionen und Frame-Shifts dazu führen, dass ein für eine Aminosäure codierendes Basentriplett zu einem Stop-Codon umgewandelt wird (nonsense-Mutation) oder ein Stop-Codon durch die Mutation für eine Aminosäure codiert (readtrough-Mutation). \\

Hinsichtlich der Eigenschaften des resultierenden Proteins unterscheidet man im Zusammenhang mit Mutationen zudem zwischen sogenannten loss-of-function- und gain-of-function-Mutationen. Loss-of-function-Mutationen führen zu einer verringerten Funktionalität oder dem vollständigen Funktionsverlust des Proteins. Gain-of-function-Mutationen bewirken dagegen eine verstärkte oder veränderte Aktivität bzw. Funktionalität des Proteins. \\

Wie bereits erwähnt führen aber nicht alle Veränderungen der DNA-Sequenz zu Störungen der Genfunktion. Veränderungen ohne unmittelbaren Krankheitswert werden als genetische Varianten bezeichnet, wenn sie innerhalb einer Spezies vermehrt auftreten ~\cite{vignal_2002, sachidanandam_2001}. Am häufigsten finden sich dabei Varianten einzelner Basenpaare, sogenannte SNP‘s (single nucleotide polymorphisms). SNP‘s kommen sowohl in codierenden als auch nicht-codierenden DNA-Abschnitten vor und treten regionsabhängig in unterschiedlicher Häufigkeit auf. SNP‘s können als genetische Marker benutzt werden ~\cite{kruglyak_1997, kwok_2003}, ihr Auftreten und ihre Verteilung spielen vor allem in der Populationsgenetik eine wichtige Rolle. Sie können Aufschluss hinsichtlich der Diversität, Selektion und Demographie einer Population geben ~\cite{nielsen_2004, shriver_2004, akey_2002}. \\

Während große strukturelle Chromosomenaberrationen unter Umständen bereits lichtmikroskopisch erkennbar sind, ist bei Punktmutationen oder SNP‘s lediglich eine einzige Base verändert. Solche Veränderungen lassen durch verschiedene molekulargenetische Verfahren detektieren ~\cite{kwok_2003, wang_1998}. Insbesondere die direkte Analyse der DNA-Sequenz mittels Sequenzierung  (siehe Kap. \ref{subsec:sanger}) ist durch die Entwicklung der sogenannten Next-Generation-Sequenzing-Verfahren (NGS) im Hochdurchsatzverfahren und mit hoher Parallelisierung durchführbar (siehe Kap. \ref{subsec:ngs}). Diese Techniken ermöglichen inzwischen umfangreiche, genomweite Analysen hinsichtlich einer großen Vielfalt molekulargenetischer Fragestellungen. Die vorliegende Arbeit befasst sich mit der Analyse und Auswertung von RAD-Sequencing-Daten. Auch die RAD-Sequenzierung gehört zu den NGS-Verfahren und dient insbesondere der Detektion von SNPs und kleinen Indels. Daher wird dieses Verfahren in Kap. \ref{sec:rad} detaillierter vorgestellt.\\

\section{Molekulargenetische Verfahren und Techniken} \label{sec:methods}
\subsection{Sanger-Sequenzierung} \label{subsec:sanger}

Nach der Erforschung der DNA-Struktur war schließlich die Entwicklung der Sanger-Sequenzierung im Jahr $1975$ ein entscheidender Meilenstein der molekulargenetischen Forschung ~\cite{sanger_1975}. Durch sie war es erstmals möglich die genaue Basensequenz eines DNA-Strangs zu bestimmen. \\

Hierbei wird die zu sequenzierende DNA-Probe in vier Teile aufgeteilt, denen jeweils eine der vier DNA-Basen in Form von radioaktiv markierten synthetischen Nukleotiden, sowie anteilig einige modifizierte Nukleotide dieser Base hinzugefügt werden. Die jeweils anderen drei Basen werden als unmarkierte und unmodifizierte Nukleotide hinzugegeben. In jedem Probengemisch ist also eine andere Base markiert und zum Teil auch modifiziert. \\

Wie bei der natürlichen DNA-Replikation (siehe Kap. \ref{subsec:replication}) während der Zellteilung kann die Proben-DNA durch Hinzugabe der DNA-Polymerase I kopiert werden. Dabei werden auch die radioaktiv markierten Nukleotide in den kopierten Strang eingebaut. Die Kopiervorgänge starten jeweils an einem kleinen Fragment mit bekannter DNA-Sequenz, dem sog. Primer. Auch die Primer werden vorab den Probengemischen beigefügt. Der Primer bindet komplementär an die Ausgangs-DNA der Probe und  ermöglicht dadurch schließlich die Bindung der DNA-Polymerase. Diese fährt vom Primer aus am Ausgangsstrang entlang und fügt dabei zu jeder Base des Ausgangsstrangs ein Nukleotid mit komplementärer Base an die Kopie an. Wird dabei eines der modifizierten Nukleotide eingefügt, so kann im nächsten Schritt kein weiteres Nukleotid mehr an den kopierten Strang angefügt werden und der Synthesevorgang wird abgebrochen. Dadurch entstehen multiple, radioaktiv markierte DNA-Fragmente unterschiedlicher Länge. In jedem der Probenansätze enden diese Fragmente mit der selben Base, da nur eine der vier Basen in modifizierter Form hinzugegeben wurde. \\

Die vier Proben werden nun nebeneinander auf ein Gel aufgetragen. Da DNA negativ geladen ist, bewegt sie sich im elektrischen Feld zur Anode. Wird also an das Gel ein elektrisches Feld angelegt, so werden die DNA-Fragmente durch das Gel bewegt. Kleinere Fragmente werden dabei schneller bewegt als größere. Dadurch ist es möglich die DNA-Fragmente der Proben entsprechend ihrer Länge aufzutrennen. Es entstehen im Gel Anhäufungen von Fragmenten gleicher Länge, die auch als Banden bezeichnet werden. Die radioaktive Markierung der Banden kann auf Röntgenfolie sichtbar gemacht werden. Bei moderneren Verfahren ist die Markierung mit radioaktiven Isotopen durch Fluoreszenzfarbstoffe abgelöst worden. Da bekannt ist in welchen Proben welche der Basen markiert ist, kann die DNA-Sequenz direkt aus der aufsteigenden Länge der DNA-Fragmente an den Banden abgelesen werden. \\

\subsection{PCR}  \label{subsec:pcr}

Zunächst waren für die Sanger-Sequenzierung große Mengen an Zellmaterial notwendig, um daraus ausreichend DNA für sichtbare Banden auf dem Gel extrahieren zu können. Die Sequenzierung mit nur geringen DNA-Mengen war nicht möglich. Durch die Entwicklung des Verfahrens der Polymerasekettenreaktion (PCR, polymerase chain reaction) ~\cite{mullis_1986} gelang es schließlich, aus einzelnen DNA-Abschnitten multiple Kopien herzustellen, so dass auch kleinste DNA-Proben der Analyse zugänglich wurden. In modernen Verfahren ist es dadurch inzwischen möglich an der DNA einer einzigen Zelle umfangreiche Analysen durchzuführen \cite{gawad_2016}. \\

Auf den bereits beschriebenen Prozessen zur Herstellung einer Kopie mit modifizierten Nukleotiden bei der Sanger-Sequenzierung basiert auch die Herstellung vielfacher, jedoch unmodifizierter Kopien bei der PCR. Temperaturabhängig wechseln hierbei mehrere Zyklen von Aufspaltung des DNA-Doppelstrangs, Primeranlagerung und DNA-Synthese ab. Entscheidend hierbei war die Verwendung einer thermostabilen DNA-Polymerase (Taq-Polymerase) \cite{green_2019}. Dadurch war es möglich mehrfach Zyklen wechselnder Temperaturen nacheinander durchzuführen, ohne dass die für die Synthese notwendige Polymerase bei höheren Temperaturen zerstört wurde. Die verwendeten Primer binden auf beiden Strängen am 3'-Ende der zu amplifizierenden DNA-Sequenz, so dass die DNA-Synthese auf beiden Strängen in 5'-3'-Richtung erfolgen kann. Im Gegensatz zur natürlichen Replikation entstehen somit keine Okazaki-Fragmente. Pro PCR-Zyklus $c$ verdoppelt sich die durch die Primer eingegrenzte DNA-Sequenz, so dass die Anzahl der Kopien exponentiell mit $2^c$ ansteigt. \\

Durch die PCR war es nun möglich verstärkt Sequenzierungen durchzuführen und so die  Erbinformation verschiedener Spezies, allen voran das Genom des Menschen im Rahmen des Human Genome Projects \cite{mcpherson_2001}, zu sequenzieren und zu kartieren. Es entstanden große Gendatenbanken (z.B. Ensembl genome database \cite{flicek_2008}, UCSC Genome Browser), welche die Referenzgenome vieler Spezies beinhalten.

\section{Next Generation Sequencing} \label{subsec:ngs}

Durch die PCR wurde also der Weg bereitet, immer umfangreichere DNA-Sequenzierungen durchführen zu können und so wurden die Sequenzierungsverfahren in den darauffolgenden Jahren zunehmend optimiert und parallelisiert \cite{shendure_2008}. Mit den heutigen Next-Generation-Sequencing-Methoden \cite{rizzo_2012, ambardar_2016} können auf geringsten DNA-Mengen kostengünstig und im Hochdurchsatzverfahren Sequenzierungen durchgeführt werden, die auch ohne vorherige spezifische Kenntnisse über die zu sequenzierenden DNA-Bereiche möglich ist \cite{dijk_2014} . Hierzu gehört unter anderem auch die RAD-Sequenzierung.

\subsection{RAD-Sequencing} \label{sec:rad}

Die RAD-Sequenzierung findet vor allem im Bereich der Populationsgenetik, Ökologie, Genotypisierung und Evolutionsforschung Anwendung. Das Prinzip der RAD-Sequenzierung basiert ursprünglich auf der Genotypisierung durch RAD-Marker \cite{miller_2007}. Dies ermöglichte das gleichzeitige Mapping natürlicher Varianten und induzierter Mutationen bei verschiedensten Organismen. RAD-Marker oder RAD-tags sind kleine DNA-Fragmente, die durch Verdau der DNA durch sogenannte Restriktionsenzyme entstehen. \\

Restriktionsenzyme sind molekulare Scheren, die in der Lage sind, die DNA an einer spezifischen DNA-Sequenz zu schneiden. Je nach verwendetem Restriktionsenzym können die Schnittstellen bezüglich des DNA-Doppelstrangs glatt oder versetzt sein \cite{roberts_2003}. Bei glatten Schnittstellen erfolgt die Aufspaltung beider DNA-Stränge an gleicher Position, so dass die entstehenden DNA-Fragmente keine Überhänge besitzen. Bei versetzten Schnittstellen erfolgt die Auftrennung auf beiden DNA-Strängen an verschiedenen Positionen, so dass bei den resultierenden DNA-Fragmenten einer der beiden Stränge gegenüber dem anderen länger hervorsteht. Diese überstehenden Sequenzen an den Fragmentenden werden häufig auch als sticky ends bezeichnet, da an ihnen besonders gut andere Sequenzen angefügt werden können. So werden beispielsweise für Sequenzierungen häufig Adaptersequenzen zur Bindung an Primer der Sequenzierplatte benötigt \cite{mardis_2017}. \\

Neben der Nutzung der entstandenen DNA-Fragmente als Marker erkannte man auch bald die Vorteile ihrer Sequenzierung \cite{baird_2008}. Aufgrund der bekannten Sequenz der Schnittstellen der Restriktionsenzyme können die Adaptersequenzen insbesondere bei sticky ends problemlos an die DNA-Fragmente gebunden werden. Die Adapter werden zusätzlich mit Barcodesequenzen versehen, welche die untersuchten Individuen identifizieren. Dadurch ist es möglich gepoolte Proben mit DNA-Fragmenten verschiedener Individuen gleichzeitig zu sequenzieren. Die dabei ausgelesenen kurzen DNA-Fragmente (Reads) lassen sich dann über ihre Barcodesequenzen den einzelnen Individuen wieder zuordnen \cite{davey_2010}. Die gleichzeitige Analyse von Probengemischen verschiedener Individuen reduziert die Kosten und den Zeitaufwand der Sequenzierung erheblich. Insbesondere führt sie aber dazu, dass die Sequenzierung bei allen Individuen unter gleichen Versuchsbedingungen stattfindet, so dass die Reads der Individuen zuverlässiger mit einander vergleichbar sind. Durch die Sequenzspezifität der Restriktionsenzyme stammen die DNA-Fragmente bei evolutionär eng beieinander liegenden oder gleichen Spezies in der Regel vom gleichen genomischen Ort (Locus). Das Vorhandensein eines Referenzgenoms (siehe Kap. \ref{subsec:pcr}) ist hierbei keine Voraussetzung mehr, vielmehr können durch statistische Analysen für jedes Individuum mögliche Loci bestimmt \cite{leggett_2014} und hinsichtlich Varianten und Mutationen interindividuell für Diversitätsanalysen verwendet werden. Dies ermöglicht auch Untersuchungen von Spezies mit unbekanntem Genom. Auch das hier implementierte Tool NodeRAD \cite{noderad} dient der Identifizierung der Loci und des Genotyps der einzelnen Individuen (Kap. \ref{chapter:kap2}) und kann in abgewandelter Form auch für den Vergleich verschiedener Individuen untereinander genutzt werden (Kap. \ref{sec:ausblick}).  \\

Für die Sequenzierung werden nur kleine Fragmente von wenigen hundert Basenpaaren Länge verwendet, größere Fragmente werden aussortiert. Dadurch kommt es zu deutlichen Verlusten bei den DNA-Fragmenten. Der Schritt der Größenselektion kann jedoch durch die Verwendung von zwei verschiedenen Restriktionsenzymen verbessert werden \cite{peterson_2012}. Bei diesem auch als ddRADSeq (double digest RADSeq) bezeichneten Verfahren werden die durch ein Restriktionsenzym erzeugten Fragmente an ihren Enden mit Adaptern versehen und anschließend ein weiteres Mal mit einem anderen Restriktionenzym behandelt. Die neu entstandenen freien Enden werden mit anderen Adapter belegt. Dadurch werden die Fragmente weiter verkleinert, so dass mehr Fragmente nach der Größenselektion zur Verfügung stehen und sequenziert werden können. Hierbei können Restriktionsenzyme, die häufig im Genom vorkommende Sequenzen schneiden, mit Restriktionsenzymen kombiniert werden, die spezifisch für seltenere Sequenzen sind. Dadurch lässt sich eine bessere Steuerbarkeit und  höhere Genauigkeit des Verfahrens erreichen. \\

Die DNA-Fragmente stammen aus dem gesamten Genom, ohne dieses vollständig abzudecken. Jedes der Fragmente wird in der Regel mehrfach sequenziert, so dass von jedem Abschnitt multiple, identische Reads entstehen. Dabei können in Abhängigkeit vom gewählten NGS-Verfahren die Reads von nur einem Ende (single-end) oder von beiden Enden aus (paired-end) sequenziert werden \cite{mardis_2017, rizzo_2012, ambardar_2016}. Durch die höhere Datendichte bei der paired-end Sequenzierung lassen sich Überlappungen der Reads und Sequenzierfehler besser erkennen. \\

Wie bereits erwähnt, stammen die Reads von verschiedenen Loci des Genoms. An einem Locus können innerhalb eines Individuums Unterschiede in der DNA-Sequenz der Reads vorkommen. Ursache hierfür ist die Ploidie des Chromosomensatzes. Der Mensch und die meisten Tierarten sind diploid, d.h. es liegt ein zweifacher Chromosomensatz vor. Jedes Chromosom ist in jeder Zelle doppelt vorhanden, wobei jeweils eines vom väterlichen und eines vom mütterlichen Elternteil stammt (homologe Chromosomen). Die homologen Chromosomen können jedoch Unterschiede in der DNA-Sequenz aufweisen, so dass am gleichen Locus unterschiedliche Varianten auf den homologen Chromosomen vorliegen. Solche Varianten eines Locus werden auch als Allele bezeichnet. Über die verschiedenen Allele kann der Genotyp bestimmt werden. Hierbei unterscheidet man Homo- und Heterozygotie. Bei  Homozygotie bezüglich eines Locus weisen die homologen Chromosomen an dieser Stelle das gleiche Allel auf, bei Heterozygotie liegen dagegen unterschiedliche Allele vor. Die Analyse der RADSeq-Daten durch NodeRAD ist für diploide Organismen vorgesehen. Bei einigen wenigen Tierarten und häufiger bei Pflanzen können auch mehr als nur zwei Chromosomensätze vorkommen (Polyploidie). Um die Analyse polyploider Organismen zu ermöglichen, sind weitere Anpassungen am zugrunde liegenden Model von NodeRAD notwendig (vgl. Kap. \ref{sec:ausblick}).

\section{Sequenzierungsdaten und -formate}

Für die Speicherung und Verarbeitung der Daten aus genetischen Analysen haben sich eine Vielzahl verschiedener Formate für verschiedenste Anwendungsfälle etabliert. Daher soll im Folgenden kurz auf die in dieser Arbeit verwendeten Formate und ihre Besonderheiten eingegangen werden.

\subsection{Reads} \label{subsec:fastq}

Die bei der Sequenzierung ausgelesenen kurzen Nukleotidsequenzabschnitte werden als Reads bezeichnet. Sie werden bei NGS-Verfahren in der Regel im FASTQ-Format gespeichert. Für jeden Read sind dort vier Zeilen vorgesehen, wobei die erste Zeile Angaben zur Identifikation des Reads und ggf. eine Beschreibung enthält. In der zweiten Zeile befindet sich die bei der Sequenzierung ausgelesene Nukleotidsequenz. Die dritte Zeile kann für weitere optionale Angaben und Beschreibungen verwendet werden. Und in der letzten Zeile wird der Quality-String hinterlegt, der für jede Base der Readsequenz Angaben zur Qualität der Sequenzierung beinhaltet. 

\subsection{Sequenzalignments} \label{subsec:samformat}

Häufig werden für die Analyse von Sequenzierungsdaten sogenannte Alignments erstellt, bei denen eine Readsequenz (Query) gegen eine andere Sequenz (Reference) verglichen wird. In Abhängigkeit von ihren Übereinstimmungen (Matches) und Unterschieden (Mismatches) werden die Sequenzen einander zugeordnet. Ein solches Alignment im SAM/BAM-Format wird auch in der vorliegenden Arbeit zur weiterführenden Analyse verwendet. \\

Das SAM/BAM-Format enthält unter anderem die ID's der Query- und Reference-Sequenzen, Informationen zur Basenqualität, Readlänge und -sequenz, den sogenannten CIGAR-String sowie verschiedene optionale Tags~\cite{sam_bam, li_2009}, wie beispielsweise den NM-Tag, der die Edit-Distanz angibt (vgl. Kap. \ref{subsec:sol_graph}). Die Edit-Distanz ist die mindestens notwendige Anzahl von Ersetzungs-, Einfügungs- und Löschungsoperationen, um die Sequenz des Source-Knotens in die Sequenz des Target-Knotens zu transformieren. Auf DNA-Ebene entspricht dies den Punktmutationen im Sinne von Substitutionen, Insertionen und Deletionen (vgl. Kap. \ref{subsec:mutation}). Der CIGAR-String hingegen ist eine kondensierte Darstellung der Unterschiede zwischen Query- und Reference-Sequenz. Im SAM-Format werden darin Matches mit $ M $ oder $ = $, Mismatches mit $ X $  oder $ S $,  Insertionen mit $ I $ und Deletionen mit $ D $ sowie jeweils mit der Anzahl der betroffenen Basen codiert (siehe auch Kap. \ref{subsec:mutation}). Aus ihrer Summe mit Ausnahme und abzüglich der Deletionen ergibt sich die Readlänge. So ergibt beispielsweise der CIGAR-String $ 69=1X24= $ ein Matching der ersten $ 69 $ Basen des Queryreads beim Abgleich mit dem Referenzread, anschließend ist bei einer Base ein Mismatch aufgetreten und schließlich folgen $ 24 $ Basen die auf den Referenzread matchen. Die Readlänge beträgt somit insgesamt $ 94 $ Basen und die Edit-Distanz im NM-Tag besitzt den Wert $ 1 $. \\

Daneben gibt es weitere Darstellungsmöglichkeiten von Veränderungen der Querysequenz gegenüber der Referenzsequenz. So können im cs-Tag des SAM-Formats verkürzt (short) oder vollständig mit Angabe der gesamten Readsequenz die genauen Veränderungen mit Bezeichnung der betreffenden Basen (long) angegeben werden. Bezüglich des oben genannten Beispiels kann der dazugehörige short cs-Tag $ 69*tc:24 $ lauten und zeigt damit an, dass nach Base $69$ ein Basenaustausch von Thymin gegen Cytosin erfolgt ist.\\

Beim BAM-Format handelt es sich um eine komprimierte Binärdatei \cite{sam_bam} mit der gleichen Information wie sie in einer SAM-Datei enthalten ist. Mit gängigen Tools, wie beispielsweise Samtools-view ~\cite{li_2009}, können beide Formate in einander umgewandelt werden.

\subsection{Varianten} \label{subsec:vcformat}
Das Variant Call Format (VCF, \cite{danecek_2011}) ist das Standardformat für die Speicherung von Daten genetischer Varianten. Neben einem Header, in dem sich Metadaten sowie Beschreibungen, Filter und Spezifikationen einzelner Spalten befinden können, gibt es tabulatorgetrennt acht erforderliche  Spalten (CHROM, POS, ID, REF, ALT, QUAL, FILTER, INFO) sowie die optionale Spalte FORMAT und beliebig viele optionale Spalten für die einzelnen Proben. Die Spalte CHROM gibt die Bezeichnung des Chromosoms an, auf dem sich die betreffende Variante befindet. In der Spalte POS wird die Position der ersten Base der Variante angegeben. Die Spalte ID kann weitere Identifikatoren enthalten. In REF wird die Referenzsequenz und in ALT eine alternative Sequenz hinterlegt. Diese beiden Spalten spezifizieren somit die Sequenzunterschiede einer Variante. Angaben zur Qualität der detektierten Varianten finden sich in der Spalte QUAL. Angewendete Filter werden in der Spalte FILTER aufgeführt. Unter der Spalte INFO können zudem weitere Eigenschaften der Variante in Form von Key-Value-Paaren hinterlegt werden, die sich auf sämtliche Probenspalten beziehen. In der Spalte FORMAT können ein oder mehrere Flags gesetzt werden, die in den Probenspalten für jede Probe individuell angegeben werden. So zeigt beispielsweise das Flag ``GT'' an, dass in den Probenspalten Angaben zum Genotyp eingetragen werden.
\let\cleardoublepage\clearpage