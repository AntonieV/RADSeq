% kapitel2.tex
\chapter{Analyse von RAD-Seq-Daten} \label{chapter:kap2}
\section{Problemstellung} \label{sec:probl}
gegeben: reads einer sample \\
Ziel: Zuordnung der Reads zu loci -> Ausgabe: Sequenz der Loci \\
es soll die Menge von Loci gefunden werden, die die Beobachtungen erklären können \\
Ergebnis: loci aus den beobachtungen der reads
im Gegensatz zu Stacks ohne Vorwissen abschätzbar, Stacks berücksichtigt keine Ploidie und Heterozygotie \\


\section{Formale Definition} \label{sec:formal}
Problem wird aufgeteilt in Teilprobleme -> Zusammenhangskomponeneten \\
für jede Komponente: Lösung finden für eine Komposition von Loci, die am besten die beobachteten Reads erklären können
-> Berücksichtigung der Basenqualität $q_{r}$ => Sequenzierfehler fließen mit geringerer liklihood in die weitere Berechnung ein -> werden weniger berücksichtigt und kommen seltener vor, z.T als noise entfernt-> loci likelihood wird mit den sequenzen mit seq-fehlern geringer sein -> seq-fehler gefiltert, da es nicht möglich ist eine  bessere lh für z.B. 3 Allele zu bekommen (von denen eines aber auf einem seq-fehler beruht), wenn es nur 2 tatsächliche Allele in einer komp gibt, d.h. lh für 2 Allele ist in diesem fall höher als für 3 -> loci likelihood enthält mutationen \\

\section{Lösungsansatz} \label{sec:solution}
Defs von Sequ, Mutationsrate, ploidie, heterozyg
\subsection{Graph und Zusammenhangskomponenten} \label{subsec:sol_graph}
Graph als Gesamtproblem -> Verkleinern durch Zerlegung in Teilprobleme -> Menge der conn comp\\
unklar wie viele loci jede conn comp beinhaltet, jede conn comp kann auch mehr als einen locus enthalten\\
\subsection{Pair Hidden Markov Model} \label{subsec:sol_phmm}
paarweises Alignment aus Minimap2 als Teilproblem -> fast jede Beobachtung wird berücksichtigt\\
Bewertung/wslk ob ein read aus einem anderen read entstanden ist\\
\subsection{Die maximale Likelihood der Allele} \label{subsec:sol_allele_lh}
\noindent0. noise nicht als Kandidaten-Allele -> liste lexicogra. sortierter Allele\\

\noindent1. lh zwischen 1 read und 1 allel; likelihood im pair: Wslk, dasss von 1 bestimmten Allel $a_{i}$ mit Fehlerrate $\epsilon$ tatsächlich die readsequenz $s_{r}$ stammt \\
$s_{r}$ Sequenz von Read $r$, $q_{r}$ Basenqualität aller Basen von Read $r$, $a_{i}$ Allel $i$ aus der Menge der Kandidatenallele $A=(a_{1},\dots, a_{n})$ mit Anzahl $n$ und $\epsilon$ Sequenzierfehlerrate
\begin{equation} \label{eqn:2-xxx1}
\tag{2-xxx1}
Pr(T=s_{r} \, | \, S=a_{i}, \epsilon) = pairHMM_{\epsilon,q_{r}}(a_{i}, s_{r})
\end{equation}

\noindent2. Info über alle vafs aller Kandidatenallele für 1 read; Zusammenhangskomponenten kann auch >1 locus enthalten -> Fractions finden in der möglichen Menge von loci \\
Berechnung der Wslk, einen Read zu beobachten anhand der gegebenen Allele-Fraktion\\
die Wslk der Fraktion $\Theta_{i}$ muss sich aus den tatsächlichen Beobachtungen ergeben => Urnenmodell (entspricht binominal model)\\
Urnenmodell: bsp. 2 farben in unterschiedl. Menge -> Wslk, wenn n Kugeln gezogen werden, dass dabei jeweils genau 50\% von jeder Farbe gezogen wurden (Binomial Formel), Anzahl der roten Kugeln P (Erfolgswslk) $\Theta_{i}$ und  > 2 Farben => Mulitnomialverteilung $\Theta_{i}$ + individuellen Verteilung $Pr$: $\Theta_{i} * Pr$
es gilt $\sum \Theta_{i} = 1$ => die mögliche Zuordnung entspricht der Häufigkeit die gezogen wurde $\Theta_{i}$ \\
Wslk. $ s_{r} $ zu beobachten unter der gegebenen Verteilung aus den Fraktionen; Wslk read zu sehen gegeben das Allel, nur ausgehend von Sequenzierfehlern \\
$\Theta=\theta_{1},\dots,\theta_{n}) \in [0,1]^n $ sind Allel-Fraktionen, $n$ ist Anzahl der Allele \\
\begin{equation} \label{eqn:2-xxx2}
\tag{2-xxx2}
Pr(s_{r} \, | \, \Theta=\theta_{1},\dots,\theta_{n}) = \sum_{i=1}^{n}\theta_{i} \, \cdotp Pr(T=s_{r} \, | \, S=a_{i}, \epsilon)
\end{equation}
\noindent3. Für alle Reads: \\
1 read $s_{r}$ mit $n$ candidaten allelen -> $n$ mögliche Allele von denen $s_{r}$ stammen kann -> es wird für jedes der allele wslk berechnet, dass die allele aus $s_{r}$ entstanden sind\\
bsp. allel i hat frac 0.25, also 1/4 für i, allel i+1 mit 0.75 => zu 25\% allel i und zu 75\% allel i+1 -> über max lh wird bestimmt, wie die seq der allele zusammenpassen -> modellierung ähnlicher,  unsicherheit bei der zuordung der reads, wenn sie weder der dem eine noch dem anderen allel entstammen, z.b. bei 0.5/0.5 => Unsicherheit wird an schritt 3 weitergegeben \\
Produkt über alle reads für jede mögliche Kombination an fractions \\
max wählen innerhalb der conn. comp\\
insgesamt:\\
=> $\Theta$ berechnen -> max lh aus allel-fractions\\
=> alles was an Unterschieden vorkommt, kann nur ein Sequenzierfehler sein\\
$m$ ist Anzahl der Reads, $L \in \{l_{i} \in \mathds{N}_{\leq n}^\phi \, | \, i=1, \dots, g\}$ ist Menge der Loci, $D = (s_{1}, \dots, s_{m}) \in \{A,C, G, T\}^{k^m}$ ist Menge der Readsequenzen/der Reads,
\begin{equation} \label{eqn:2-xxx3}
\tag{2-xxx3}
L(\Theta=\theta_{1},\dots,\theta_{n} \, | \, D) = Pr(D \, | \, \Theta=\theta_{1},\dots,\theta_{n}) = \prod_{r=1}^{m}Pr (s_{r} \, | \, \Theta=\theta_{1},\dots,\theta_{n})
\end{equation}
für jede vaf wurde also liklihood über alle reads bestimmt -> Maximum 
\subsection{Die maximale Likelihood der Loci}
connected comp in loci aufsplitten\\
Zuordnung zu loci, also welche genomischen loci stecken dahinter, wie werden die allele den loci zugeordnet -> max parximony, prinzip der einfachsen möglichen lösung \\
-> ploidy als eingabeparam -> zuordnung der candidaten allele zu loci\\
lh ausrechnen aus anzahl der allele und loci -> gerade die loci, die die seq. der reads am besten unter der geg. ploidy und heterozygotie erklären haben max lh
max\_lh\_vafs nur noch 1 Vektor\\
Zuordnung von cand-Allelen zu den loci => die beobachteten max\_lh\_allel\_fractionen müssen sie durch die gewählten loci erklären, z.B. 2 loci => 1.locus: allel 0, 0 homozyg, 2. locus allel 0, 2 heterozygot\\
-> bei z.B 3 Allelen sind min 2 Loci notwendig => min \# der loci, so dass alle ausgewählten allele passen -> diese menge muss gefunden werden und muss wieder max lh haben\\
unter der annahme der loci-verteilung muss bestimmt werden ob sinnvoll, d.h allel-vafs müssen sich zu 1 summieren -> indikator-funktion: wenn 1, dann gilt Pr(T|S), d.h. Wslk, dass sich aus S die seq T gebildet hat\\

indicator-function $z_{l} \in {[0,1]}$: wird 1, wenn anzahl des auftretens aller allele $A = (a_{1}, \dots, a_{n})$ in einer möglichen locus-verteilung $l = (l_{1}, \dots, l_{g \, \cdotp \phi})$ jeweils genau ihrer absoluten Häufigkeit aus der in Kap. \ref{subsec:sol_allele_lh} errechneten vaf mit maximaler Likelihood entspricht, also $\theta_{i} \, \cdotp g \, \cdotp \phi$ entsprechen, andernfalls gilt $ z_{l} = 0 $. Seien dabei  $\Theta=\theta_{1},\dots,\theta_{n}) \in [0,1]^n $ die vaf mit maximaler Likelihood, $ \phi $ die Ploidie und gilt für eine mögliche Locusverteilung $l$ außerdem $L \in \{l_{j} \in \mathds{N}_{\leq n}^\phi \, | \, j=1, \dots, g\}$, so lässt sich die Indikatorfunktion wie folgt definieren:
\begin{equation} \label{eqn:2-xxx4}
\tag{2-xxx4}
z_{l}=\prod_{i=1}^{n}1_{\sum_{j=1}^{g}\sum_{k=1}^{\phi}1_{l_{j,k}=i} = \theta_{i} \, \cdotp g \, \cdotp \phi}
\end{equation}

pHMM: Wslk, dass allel 2 $ a_{l_{j,2}} $ aus allel 1 $ a_{l_{j,1}} $ entstanden ist -> heterozygotie $\eta$ (in conf konfigurierbar) =>$ match = 1 - (het_{sub} + het_{del} + het_{ins}) $
\begin{equation} \label{eqn:2-xxx5}
\tag{2-xxx5}
Pr(T=a_{l_{j,2}} \, | \, S=a_{l_{j,1}}, \eta) = pairHMM_{\eta}(a_{l_{j,1}}, a_{l_{j,2}})
\end{equation}

Kombi von loci -> lh berechnen
\begin{equation} \label{eqn:2-xxx6}
\tag{2-xxx6}
Pr(\Theta,A \, | \, L=\{l_{j}\, |\, j=1,\dots,g\})=z_{l} \, \cdotp \prod_{j=1}^{g}Pr(T=a_{l_{j,2}} \, | \, S=a_{l_{j,1}})
\end{equation}

=> maximum der loci Likelihoods wählen
\subsection{Varianten und Genotyp}
mit den allelen $A = (a_{1}, \dots, a_{n})$ jedes locus $L \in \{l_{j} \in \mathds{N}_{\leq n}^\phi \, | \, j=1, \dots, g\}$, allelsequenz, genotyp\\
Menge von Loci, die die Beobachtungen erklären