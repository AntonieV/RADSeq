% kapitel2.tex
\chapter{Analyse von RAD-Seq-Daten} \label{chapter:kap2}
\section{Problemstellung} \label{sec:probl}
gegeben: reads einer sample \\
Ziel: Zuordnung der Reads zu loci -> Ausgabe: Sequenz der Loci \\
es soll die Menge von Loci gefunden werden, die die Beobachtungen erklären können \\
Ergebnis: loci aus den beobachtungen der reads
im Gegensatz zu Stacks ohne Vorwissen abschätzbar, Stacks berücksichtigt keine Ploidie und Heterozygotie \\


\section{Formale Definition} \label{sec:formal}
Problem wird aufgeteilt in Teilprobleme -> Zusammenhangskomponeneten \\
für jede Komponente: Lösung finden für eine Komposition von Loci, die am besten die beobachteten Reads erklären können
-> Berücksichtigung der Basenqualität $q_{r}$ => Sequenzierfehler fließen mit geringerer liklihood in die weitere Berechnung ein -> werden weniger berücksichtigt und kommen seltener vor, z.T als noise entfernt-> loci likelihood wird mit den sequenzen mit seq-fehlern geringer sein -> seq-fehler gefiltert, da es nicht möglich ist eine  bessere lh für z.B. 3 Allele zu bekommen (von denen eines aber auf einem seq-fehler beruht), wenn es nur 2 tatsächliche Allele in einer komp gibt, d.h. lh für 2 Allele ist in diesem fall höher als für 3 -> loci likelihood enthält mutationen \\

\section{Lösungsansatz} \label{sec:solution}
Defs von Sequ, Mutationsrate, ploidie, heterozyg
\subsection{Graph und Zusammenhangskomponenten} \label{subsec:sol_graph}
Graph als Gesamtproblem -> Verkleinern durch Zerlegung in Teilprobleme -> Menge der conn comp\\
unklar wie viele loci jede conn comp beinhaltet, jede conn comp kann auch mehr als einen locus enthalten\\
\subsection{Pair Hidden Markov Model} \label{subsec:sol_phmm}
paarweises Alignment aus Minimap2 als Teilproblem -> fast jede Beobachtung wird berücksichtigt\\
Bewertung/wslk ob ein read aus einem anderen read entstanden ist\\
\subsection{Die maximale Likelihood der Allele} \label{subsec:sol_allele_lh}
\noindent0. noise nicht als Kandidaten-Allele -> liste lexicogra. sortierter Allele\\

\noindent1. lh zwischen 1 read und 1 allel; likelihood im pair: Wslk, dasss von 1 bestimmten Allel $a_{i}$ mit Fehlerrate $\epsilon$ tatsächlich die readsequenz $s_{r}$ stammt \\

\noindent2. Info über alle Kandidatenallele für 1 read; Zusammenhangskomponenten kann auch >1 locus enthalten -> Fractions finden in der möglichen Menge von loci \\
Berechnung der Wslk, einen Read zu beobachten anhand der gegebenen Allele-Fraktion\\
die Wslk der Fraktion $\Theta_{i}$ muss sich aus den tatsächlichen Beobachtungen ergeben => Urnenmodell (entspricht binominal model)\\
Urnenmodell: bsp. 2 farben in unterschiedl. Menge -> Wslk, wenn n Kugeln gezogen werden, dass dabei jeweils genau 50\% von jeder Farbe gezogen wurden (Binomial Formel), Anzahl der roten Kugeln P (Erfolgswslk) $\Theta_{i}$ und  > 2 Farben => Mulitnomialverteilung $\Theta_{i}$ + individuellen Verteilung $Pr$: $\Theta_{i} * Pr$
es gilt $\sum \Theta_{i} = 1$ => die mögliche Zuordnung entspricht der Häufigkeit die gezogen wurde $\Theta_{i}$ \\
Wslk. $ s_{r} $ zu beobachten unter der gegebenen Verteilung aus den Fraktionen; Wslk read zu sehen gegeben das Allel, nur ausgehend von Sequenzierfehlern \\

\noindent3. Für alle Reads: \\
1 read $s_{r}$ mit $n$ candidaten allelen -> $n$ mögliche Allele von denen $s_{r}$ stammen kann -> es wird für jedes der allele wslk berechnet, dass die allele aus $s_{r}$ entstanden sind\\
bsp. allel i hat frac 0.25, also 1/4 für i, allel i+1 mit 0.75 => zu 25\% allel i und zu 75\% allel i+1 -> über max lh wird bestimmt, wie die seq der allele zusammenpassen -> modellierung ähnlicher,  unsicherheit bei der zuordung der reads, wenn sie weder der dem eine noch dem anderen allel entstammen, z.b. bei 0.5/0.5 => Unsicherheit wird an schritt 3 weitergegeben \\
Produkt über alle reads für jede mögliche Kombination an fractions \\
max wählen innerhalb der conn. comp\\
insgesamt:\\
=> $\Theta$ berechnen -> max lh aus allel-fractions\\
=> alles was an Unterschieden vorkommt, kann nur ein Sequenzierfehler sein\\
\subsection{Die maximale Likelihood der Loci}
connected comp in loci aufsplitten\\
Zuordnung zu loci, also welche genomischen loci stecken dahinter, wie werden die allele den loci zugeordnet -> max parximony, prinzip der einfachsen möglichen lösung \\
-> ploidy als eingabeparam -> zuordnung der candidaten allele zu loci\\
lh ausrechnen aus anzahl der allele und loci -> gerade die loci, die die seq. der reads am besten unter der geg. ploidy und heterozygotie erklären haben max lh
max\_lh\_vafs nur noch 1 Vektor\\
Zuordnung von cand-Allelen zu den loci => die beobachteten max\_lh\_allel\_fractionen müssen sie durch die gewählten loci erklären, z.B. 2 loci => 1.locus: allel 0, 0 homozyg, 2. locus allel 0, 2 heterozygot\\
-> bei z.B 3 Allelen sind min 2 Loci notwendig => min \# der loci, so dass alle ausgewählten allele passen -> diese menge muss gefunden werden und muss wieder max lh haben\\
unter der annahme der loci-verteilung muss bestimmt werden ob sinnvoll, d.h allel-vafs müssen sich zu 1 summieren -> indikator-funktion: wenn 1, dann gilt Pr(T|S), d.h. Wslk, dass sich aus S die seq T gebildet hat\\
indicator-function: wird 1, wenn anzahl des auftretens des allels = $\Theta_{i} * g * \phi$ 
Kombi von loci -> lh berechnen
pHMM: Wslk, dass allel 2 aus allel 1 entstanden ist -> heterozygotie (in conf konfigurierbar) =>$ match = 1 - (het_{sub} + het_{del} + het_{ins}) $

\subsection{Varianten und Genotyp}
mit den allelen jedes locus, allelsequenz, genotyp\\
Menge von Loci, die die Beobachtungen erklären