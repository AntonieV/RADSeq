% kapitel5.tex
\chapter{Evaluation an simulierten Datensätzen} \label{sec:}
prototyp, zunächst für ploidie 2 -> bei höherer ploidie multiples Sequenzalignment

%anderer Ansatz: bei mehr als 2 allelen -> Teilgraph aus Allelen mit edit dist als Kantengewicht -> min Spannbaum => über diese Kanten lh aus pHMM für jede Kanten lösen, summe über jede mögliche wurzel (rooted mst)
\section{} \label{sec:}
\subsection{} \label{subsec:}