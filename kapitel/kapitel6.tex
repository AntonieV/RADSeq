% kapitel6.tex
\definecolor{light-gray}{gray}{0.93}
\chapter{Ausblick} \label{sec:ausblick}

Das Tool NodeRAD \cite{noderad} wurde als Prototyp für die Analyse von single-end RADSeq-Daten diploider Organismen implementiert. Das zugrundeliegenden Model erlaubt es, ohne Kenntnis des Genoms und nur mit Hilfe der Sequenzierfehlerrate und Heterozygotiewahrscheinlichkeit die wahrscheinlichsten Allele zu identifizieren, die den beobachteten Reads zugrunde liegen. Im Anschluss wird für diese Allele die jeweils wahrscheinlichste Locikomposition ermittelt. Hierbei zeigt sich an simulierten Daten eine gute Erkennungsrate der Loci ($ 92\, \% $), wobei homo- und heterozygote Varianten problemlos gefunden werden. Mit einer Rate von $ 7.5\, \% $ werden aber auch falsch-heterozygote Loci bestimmt. Dadurch sind von allen durch NodeRAD als heterozygot ausgewiesenen Loci nur $ 33.\overline{3}\, \% $ auch tatsächlich heterozygot. Diese Statistiken sind zunächst nur eine grobe Abschätzung anhand einer kleineren simulierten Stichprobe. Für eine bessere Beurteilung sollte das Modell an größeren realen Datensätzen getestet werden. Es ist in Zukunft eine effizientere Programmierung des Tools in der Programmiersprachen Rust geplant, so dass diese umfangreicheren Tests erst im Anschluss erfolgen sollten. \\

Der dominierende Schritt hinsichtlich Laufzeit und Arbeitsspeicherbedarfs sind bei NodeRAD die Bestimmung der Allelkombinationen und die über ihre Allel-Fractions ausgeführten Berechnungen. Hier konnten durch optimale Wahl der Schwellenwerte für die Mindesthäufigkeiten der Allele unter- und oberhalb einer festgelegten Clustergröße deutliche Verbesserungen erreicht werden. Dieser Ansatz könnte weiter optimiert werden, so dass diese Werte beispielsweise direkt in Beziehung zur Clustergröße stehen. Denkbar wäre hier die Formulierung einer Funktionsgleichung, die in Abhängigkeit von der Clustergröße die maximale Anzahl der zu prüfenden Allele festlegt, wobei die Allele absteigend nach ihrer Häufigkeit als Kandidatenallele inkludiert werden. Eine andere Möglichkeit wäre, direkt rechenintensive Cluster zu identifizieren und dort eine Optimierung bei der Wahl der Kandidatenallele durchzuführen. Solche rechenintensiven Cluster können zum einen sehr große Zusammenhangskomponenten sein, zum anderen aber auch eher kleine Cluster, die aber viele verschiedene Allele jeweils mit nur geringen Häufigkeiten aufweisen. Eine explizite Festlegung von Schwellenwerten nur für diese Cluster, ermöglicht eine genauere Analyse bei den übrigen, weniger problematischen Clustern. Dies könnte auch zu einem positiven Effekt hinsichtlich der Rate falsch-heterozygoter Loci führen. \\

Wie bereits in Kap. \ref{sec:rad} beschrieben, ermöglicht die höhere Datendichte bei paired-end Reads eine größere Genauigkeit. Durch die damit verbundene Verringerung der Überlappungen der Reads könnte den Anteil falsch-heterozygoter Loci gesenkt werden. Allerdings führt die Verwendung von paired-end Reads dazu, dass heterozygote Varianten im Bereich der Restriktionsstelle nicht erkannt werden. Liegt bei heterozygoten Varianten die Mutation im Bereich der Restriktionsstelle, so kann das Restriktionsenzym dort nicht schneiden. In diesem Fall wäre es bei paired-end Sequenzierung nicht möglich, die Reads von beiden Seiten zu lesen. Dadurch würden die Reads des Allels mit der Variante verworfen werden und der Locus würde in der Analyse homozygot erscheinen. Bei single-end Reads würde das Allel mit der Variante von einer Seite gelesen werden. Da die Reads hier unabhängig von einander sind, würde nichts verworfen werden, so dass die Heterozygotie erkennbar bleibt. Aus diesem Grund wurde NodeRAD für single-end Reads konzipiert, so dass die Rate falsch-homozygoter Loci gering ist. In den simulierten Testdaten betrug diese Rate $0 \, \%$. \\

Wie bereits mehrfach angesprochen, soll in diesem Kapitel auch auf die Analyse polyploider Spezies eingegangen werden. Der Prototyp NodeRAD wurde zunächst für diploide Spezies implementiert. Die Likelihoodberechnungen für die Loci-Zuordnungen kann dadurch unter Einbeziehung der Heterozygotiewahrscheinlichkeiten über ein pairHMM erfolgen. Bei polyploiden Organismen genügt der paarweise Vergleich nicht mehr. Vielmehr muss das Alignment über mehrere Kandidatenallele erfolgen, so dass hier für die Loci-Zuordnung und Bestimmung des Genotyps Algorithmen zur Erzeugung multipler Sequenzalignments Anwendung finden müssen \cite{chowdhury_2017,bawono_2017,chatzou_2015}. Da diese im Allgemeinen eine wesentlich höhere Laufzeit aufweisen, ist die vorherige Eingrenzung der möglichen Loci-Kombinationen, wie dies bereits für diesen Prototyp implementiert wurde (vgl. Kap. \ref{subsec:comb_loci} und Satz \ref{thm:comb_indicator}), von großem Vorteil. \\

Der Ansatz von NodeRAD ermöglicht es zudem im Rahmen von Diversitätsanalysen in einer zweiten Iteration durch Erweiterung des Modells auch einen interindividuellen Vergleich durchzuführen. Wie in \autoref{fig:workflow_all} bereits angedeutet, kann aus den Ergebnissen des Workflows, d.h. aus den VCF-Files der einzelnen Individuen, erneut ein Sequenzalignment durch Minimap2 gebildet und daraus ein Graph erzeugt werden. Aus den Zusammenhangskomponenten können dann die Kandidatenallele ermittelt und die jeweils wahrscheinlichsten Allele-Fractions über allen Individuen bestimmt werden. Diese dienen im Anschluss der Zuordnung zu einem oder mehreren gemeinsamen Loci. Je mehr Varianten die resultierenden Loci aufweisen, desto besser geht es der Population, aus der die getesteten Individuen stammen.\\

\let\cleardoublepage\clearpage