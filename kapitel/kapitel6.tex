% kapitel6.tex
\chapter{Zusammenfassung und Ausblick} \label{sec:ausblick}
var callen\\
verschiedene Individuen mit einander vergleichen bei gleichem Anzatz, aber nun mit locus seq. aus den vcf's f mehrere individuen als eingabe -> vergleich der individuen \\
Diversität -> je diverser, desto besser geht es der Population\\
SE-Reads besser geeignet für Analyse: bei mutation im bereich einer der rev-schnittstelle  würde bei PE nicht geschnitten werden und der read würde nicht geclustert und ausgewertet werden -> heterozygotie nicht mehr sichtbar, bei SE sind aber die reads unabhängig von einander, die Heterozygotie an dieser stelle würde erfasst werden\\
eventuell Vgl mit ILP:\\
-> ILP wird sehr groß, die Lösung wird 2 schrittig gefunden, aber dadurch kein globales Optimum, dass für jede optimale Zuordnung diese durch erwartete Sequenziertiefe erklärt -> es könnte ebenso eine nicht optimale Lösung im ILP eine optimale Lösung in zusammenschau mit der seq-tiefe werden \\
=> Seq-tiefe ist nur sehr bedingt als Maß geeignet, z.B. Cutting-Enzym schneidet nicht richtig -> seq-tiefe und häufigkeit ändern sich\\

prototyp, zunächst für ploidie 2 -> bei höherer ploidie multiples Sequenzalignment ~\cite{liu_2010} ~\cite{durbin_1998}

\section{} \label{sec:}
\subsection{} \label{subsec:}