% kapitel3.tex
\chapter{Implementierung} \label{sec:alg}
Für das in Python implementierte RAD-Sequencing-Tool, NodeRAD, wurde zur Workflowintegration das Workflow Management System Snakemake verwendet ~\cite{koester_2012_1, koester_2012_2}. Die einzelnen Analyseschritte werden dabei über Regeln abgebildet. Für jede Regel können neben dem zu verwendenden Script oder Shell-Kommando sowie den Pfadangaben für In- und Output auch zusätzliche Optionen festgelegt werden. Dazu gehören beispielsweise Angaben zu Parametern bzw. Argumenten für die verwendeten Tools, Pfadangaben für Log-Dateien oder die Anzahl der zu verwendenden Threads. 

Als Input benötigt der Workflow eine Datei im FASTQ-Format (siehe Kap. \ref{subsec:fastq}), welche die single-end Reads der verschiedenen Individuen mit ihren Identifikationsbezeichnungen, der Basensequenz und Angaben zur  Basenqualität enthält. Des Weiteren wird eine Tabelle im tsv-Format benötigt, in der die Zuordnung der Probennamen zu den Individuen und ihren Barcode-Sequenzen definiert ist. Nach dem Preprocessing, der Qualitätskontrolle der Reads und dem Sequence-Alignment erfolgt die RAD-Seq-Analyse durch NodeRAD. Hierbei werden die Wahrscheinlichkeiten der Allele-Fractions und der möglichen Loci bestimmt. Die Loci mit der höchsten Wahrscheinlichkeit werden schließlich mit den Sequenzen ihrer Allele in einer Datei im Variant Call Format (VCF) ausgegeben.

\section{Preprocessing} \label{sec:preproc}

Im Preprocessing werden durch das Tool Cutadapt ~\cite{martin_2011} die Reads jedes Individuums anhand ihrer Barcodesequenzen identifiziert und extrahiert (Demultiplexing). Hiernach werden die Barcodesequenzen entfernt (Trimming) und die Reads jedes Individuums in separaten Dateien im FASTQ Format abgelegt. \\
Im Anschluss an das Trimming erfolgt eine Qualitätskontrolle durch das Tool FastQC  ~\cite{andrews_2012}. Dabei werden einige allgemeine Statistiken zu den Rohdaten der Reads generiert, wie beispielsweise zur Basenqualität, zum GC-Gehalt, dem Anteil an Duplikaten oder überrepräsentierten Sequenzen. Durch das Tool MultiQC ~\cite{ewels_2016} wird aus diesen Statistiken und den Log-Dateien von Cutadapt ein html-Report mit diversen Plots zur Veranschaulichung erstellt.

\section{Edit-Distanzen} \label{sec:edit}

Für die spätere Konstruktion eines Graphen basierend auf den Edit-Distanzen zwischen den Readsequenzen wird für jedes Individuum zunächst ein Sequenzalignment (vgl. Kap. \ref {subsec:samformat}) mit Hilfe des Tools Minimap2 ~\cite{li_2018} erstellt. Hierbei werden alle Readsquenzen paarweise mit einander verglichen. Das Ergebnis des Mappings wird im SAM/BAM-Format ~\cite{li_2009} ausgegeben und enthält neben den Angaben zur Query- und Referenzsequenz auch den CIGAR-String sowie den NM-Tag mit den Edit-Distanzen. Der CIGAR-String und der NM-Tag definieren wichtige Kanteneigenschaften des späteren Graphen. \\

\section{Konstruktion des Graphen} \label{sec:graph}
\subsection{Knoten des Graphen} \label{subsec:nodes}
NodeRAD benötigt als Input zu jedem Individuum die getrimmten single-end Read-Daten sowie das Sequenzalignment. Zunächst wird daraus für jedes Individuum ein eigener, gerichteter Graph $ G $ mit $ G = (V,E) $ erstellt. Seine Knoten, $ V $, werden durch die einzelnen Reads repräsentiert. Entsprechend ergeben sich die Knoteneigenschaften aus den Daten der Reads, diese werden den FASTQ-Dateien nach Ausführung von Cutadapt (siehe \ref{sec:preproc}) entnommen. Die Kanten, $ E $, zwischen den Knoten ergeben sich aus dem Vergleich ihrer Sequenzen im Rahmen des Sequenzalignments mittels Minimap2 (siehe \ref{sec:edit}).\\

Zusätzlich entnimmt NodeRAD der Konfigurationsdatei des Workflows einige Konstanten und Grenzwerte für die späteren Berechnungen. Dazu gehören die Mutationsraten und Heterozygotiewahrscheinlichkeiten für Substitutionen, Insertionen und Deletionen, die Ploidie des Chromosomensatzes der untersuchten Spezies und Grenzwerte. Die Konstanten werden als Grapheigenschaften im Graphen abgelegt. Als konfigurierbare Grenzwerte gibt es für NodeRAD einen Schwellenwert für die maximal zulässige Editierdistanz, bei dem zwei Knoten noch durch eine Kante verbunden werden sowie Schwellenwerte zum Filtern selten vorkommender Sequenzen ab einer bestimmten Clustergröße, die als Hintergrundrauschen nicht in der Berechnung Berücksichtigung finden sollen. \\

Zur Konstruktion des Graphen wird die Python-Library graph-tool ~\cite{peixoto_2014} genutzt. Die Knoten werden aus den FASTQ-Daten der getrimmten Reads mittels SeqIO aus der Library Biopython ~\cite{cock_2009_1} ausgelesen und im Graphen mit den Knoteneigenschaften ihrer Basensequenz, einer internen ID sowie Angaben zur Basenqualität abgelegt. Die Codierung des Qualitystrings der Reads variiert je nach verwendeter Platform. Daher wird er durch SeqIO ausgelesen und für jede Base in ein einheitliches Maß, den Phred Quality Score $ Q $, decodiert ~\cite{cock_2009_2}. Zusätzlich wird aus den Phred Quality Scores die geschätzte Fehlerwahrscheinlichkeit $ P $ für jede Base nach Formel \eqref{eqn:3-1} bestimmt ~\cite{ewing_1998}.  

\begin{equation} \label{eqn:3-1}
    \tag{3-1}
    P = 10^{\frac{-Q}{10}}
\end{equation}

Für jeden Knoten werden die Vektoren mit den Phred Qualitiy Scores und den geschätzen Fehlerwahrscheinlichkeiten der Basen des Reads als Knoteneigenschaften gespeichert. \\

\subsection{Kanten des Graphen} \label{subsec:edges}
Die Kanten des Graphen definieren sich durch das mittels Minimap2 erzeugten Sequenzalignments (vgl. Kap. \ref{sec:edit}). Jedes Alignment zwischen zwei Reads entspricht im Graphen einer gerichteten Kante $e = (source,\; target)$, die den Vergleich der Query- zur Referenzsequenz repräsentiert. Sie verbindet somit zwei der zuvor aus der FASTQ-Daten erzeugten Knoten. Das Auslesen des sam-Formats des Alignmentfiles erfolgt mit Hilfe der Python-Library pysam ~\cite{pysam}. Dabei wird die Edit-Distanz aus dem NM-Tag zunächst genutzt, um nur Kanten in den Graphen aufzunehmen, die bereits einen optimierten Minimap2-Path darstellen. Liegen diese unterhalb des durch die Konfigurationsdatei festgelegten Grenzwertes \lstinline|threshold_max_edit_distance|, so wird die Kante dem Graphen hinzugefügt. Dabei werden als Kanteneigenschaften die Edit-Distanz, die CIGAR-Tupel sowie die aus der Basenqualität und Mutationsrate bestimmte Likelihood hinzugefügt. Zusätzlich kann zur Kontrolle oder für eine spätere Verwendung auch der CIGAR-String selbst als Kanteneigenschaft gespeichert werden, falls bei Minimap2 die Option zur Erzeugung des cs-Tags aktiviert wurde. Die CIGAR-Tupel werden durch pysam aus dem CIGAR-String geparsed, hierbei handelt es sich um eine Liste von Tupeln, die jeweils aus Integer-Wertepaaren bestehen. Der erste Wert jedes Tupels gibt die spezifische Operation des Matches oder Mismatches. So entspricht beispielsweise ein Wert von $ 7 $ oder $ 0 $ einem Match und ein Wert von $ 2 $ einer Deletion. Der zweite Werte jedes Tupels gibt die Anzahl der Basen an, die von der entsprechenden Operation betroffen sind. \\

Diese CIGAR-Tupel werden für die Berechnung der Likelihood zwischen zwei Knoten benötigt, dies erfolgt in der Methode \lstinline|get_alignment_likelihood()| (Algorithmus \ref{alg:lh_read}) aus dem Modul \lstinline|likelihood_operations.py|. Dabei wird aus den p-Werten der Basenqualität für jede Base der Query-Sequenz die Wahrscheinlichkeit errechnet, dass es sich im Falle eines Matches um die korrekte Base handelt  \eqref{eqn:3-2} bzw. im Falle eines Mismatches, dass es sich um einen Sequenzierfehler \eqref{eqn:3-3} oder um eine Mutation handelt \eqref{eqn:3-4}. \\

Die Berechnung der Likelihoods für die paarweisen Vergleiche der Reads basiert auf dem in Kap. ~\ref{subsec:sol_phmm} beschriebenen pair Hidden Markov Model. Hierbei repräsentiert das durch Minimap2 bestimmte Sequenzalignment bereits den wahrscheinlichsten Pfad durch die pairHMM-Matrix. Da dieser Pfad ohnehin die Wahrscheinlichkeit des pairHMM dominieren würde, wird zugunsten der Laufzeit direkt auf das Alignment von Minimap2 zurückgegriffen, um die Likelihoods zwischen den Readsequenzen zu bestimmen. Dabei werden die Sequenzierfehlerrate $ \epsilon $ und Basenqualität $ q_{query} $ durch die bereits zuvor ermittelte geschätzte Fehlerrate $ p_{query} $ berücksichtigt. Die Likelihood $ pairHMM_{\epsilon, q_{query}} \;(s_{ref}\;|\; s_{query}) $, dass der Queryread aus dem Referenzread allein durch Sequenzierfehler und Mutationen entstanden ist, errechnet sich schließlich aus dem Produkt der Likelihoods $ L_{i} $ für jede Base $ b $ an jeder Position $ i $ innerhalb der Sequenz $ s $ des Queryreads $ s_{query} $ im Vergleich zum Referenzread $ s_{ref} $.
\begin{equation} \label{eqn:3-2}
\tag{3-2}
pairHMM_{\epsilon, q_{query}} \;(s_{query}\;|\; s_{ref}) = \prod_{i=1}^{k}L_{i}
\end{equation}

Jede Base $ b $ an Position $ i $ einer Readsequenz $ s $ der Länge $ k $ lässt sich also definieren als $ b \in \{\,b_{i}\in \{A,C,G,T\}^k\;,\; b_{i} \in s \;|\; i = 1, \dotsb, k \,\}$. Seien $ b_{i\,_{ref}} $ und $ b_{i\,_{query}} $ die Basen der Query- und der Referenzsequenzen an Position $ i $ einer Sequenz und $  p_{i\,_{query}} $ die geschätzte Fehlerrate von $ b_{i\,_{query}} $, die sich aus dem Phred Quality Score $ Q $ nach  \eqref{eqn:3-1} ergibt. Seien zudem $ m_{sub} $, $ m_{ins} $ und $ m_{del} $ die über die Konfigurationsdatei festgelegten Mutationsraten für Substitutionen, Insertionen und Deletionen. Dann errechnet sich die Likelihood $ L_{i} = Pr(b_{i\,_{ref}}\;|\; b_{i\,_{query}})$ an der Position $ i $ im Falle eine Matches unter Berücksichtigung der geschätzten Fehlerrate durch:
\begin{equation} \label{eqn:3-3}
\tag{3-3}
L_{i\,_{match}} = 1 - p_{i\,_{query}}
\end{equation}

Bei einem Mismatch dagegen müssen die Wahrscheinlichkeiten von Mutationen und Sequenzierfehlern berücksichtigt werden. Im Falle einer Mutation muss in die Wahrscheinlichkeit eines Matches auch die Mutationsrate des aufgetretenen Mismatches $ m_{rate} \in \{\,m_{sub},\,  m_{ins},\, m_{del}\,\} $ einbezogen werden:
\begin{equation} \label{eqn:3-4}
\tag{3-4}
L_{i\,_{mut}} = m_{rate}\; \cdotp \;(1 - p_{i\,_{query}})
\end{equation}

Die Wahrscheinlichkeit eines Sequenzierfehlers, also dass anstelle der sequenzierten Base tatsächlich eine der drei anderen Basen vorliegt, entspricht $ 1/3 $ der geschätzten Fehlerrate des Phred Quality Scores:
\begin{equation} \label{eqn:3-5}
\tag{3-5}
L_{i\,_{seqerr}} = \frac{1}{3} \; \cdotp \; p_{i\,_{query}}
\end{equation}

Aus \eqref{eqn:3-4} und \eqref{eqn:3-5} errechnet sich also die Likelihood bei einem Mismatch durch:
\begin{equation} \label{eqn:3-6}
\tag{3-6}
L_{i\,_{mismatch}} = (1-m_{rate}) \; \cdotp \; L_{seqerr} \; \cdotp \; L_{mut}
\end{equation}

Aus den Liklihoods von Matches \eqref{eqn:3-3} und Mismatches \eqref{eqn:3-4} kann somit schließlich nach \eqref{eqn:3-2} die Likelihood zwischen den Reads paarweise bestimmt werden.\\

Für eine existierende Kante, von der die CIGAR-Tupel bekannt sind, kann die Methode \lstinline|get_alignment_likelihood()| zudem die Likelihood in entgegengesetzter Richtung bestimmen. Dabei wird der Queryread als Referenzread betrachtet und umgekehrt. Dies ist über das boolsche Argument \lstinline{reverse} steuerbar. Gilt \lstinline|reverse = True|, so werden für die übergebenen CIGAR-Tupel Insertionen zu Deletionen und Deletionen zu Insertionen umbewandelt, anschließend wird die Likelihood nach \eqref{eqn:3-6} berechnet.

Zur zusätzlichen Veranschaulichung ist die Methode \lstinline|get_alignment_likelihood()| in Algorithmus \ref{alg:lh_read} in Pseudocode dargestellt.

\begin{algorithm}[H]
	\caption{Berechnung der Likelihood zwischen zwei Reads}  \label{alg:lh_read}
	\begin{algorithmic}[1]	
		\Function{get\_alignment\_likelihood}{$ m_{sub} $, $ m_{ins} $, $ m_{del} $, $ CIGAR-Tuples $, $ p_{query} $, reverse}
		\State $ likelihood \gets 1.0 $, $ index \gets 0 $
		\If {$reverse$}
		    \State swap values of $ m_{ins} $ and $ m_{del} $
	    \EndIf
	    \ForAll {$ (operation, length) \in CIGAR-Tuples $}
	    \If {$operation \in match $}
		    \While{$ index < length $}
		        \State $ likelihood\, \gets likelihood \,\cdotp (1-p_{query}[index]) $
		    	\State $ index \gets index + 1 $
		    \EndWhile
	    \EndIf
	    \If {$operation \in mismatch $}
	        \State $ m_{rate} \gets 0 $
	        \If {$operation \in substitution $}
	            \State $ m_{rate} \gets m_{sub} $
	        \EndIf
	        \If {$operation \in insertion $}
	            \State $ m_{rate} \gets m_{ins} $
	        \EndIf
	        \If {$operation \in deletion $}
	            \State $ m_{rate} \gets m_{del} $
	        \EndIf
	        \While{$ index < length $}
		        \State $ likelihood\, \gets likelihood \,\cdotp (1 - m_{rate})\,\cdotp \frac{1}{3} \,\cdotp p_{query}[index] \, +  m_{rate}\,\cdotp $         
		         \State \hspace{63pt}  $ (1 - p_{query}[index]) $ 		        
		        \State $ index \gets index + 1 $
	        \EndWhile
	    \EndIf
		\EndFor
		\State \Return $likelihood$
		\EndFunction		
	\end{algorithmic}
\end{algorithm}

Nach Abschluss der Graphkonstruktion werden für jedes Individuum noch einige Statistiken in die Log-Dateien geschrieben. Hier werden neben der Anzahl der Knoten und Kanten des Graphen auch die Anzahl der Substitutionen bzw. SNPs, Insertionen und Deletionen festgehalten, die beim Auslesen der CIGAR-Tupel registriert wurden. Zudem findet sich hier auch die maximal vorkommenden Edit-Distanz über alle Knoten, sofern sich diese unterhalb des festgelegten Schwellenwertes liegt. Ansonsten entspricht sie dem in der Konfigurationsdatei angegebenen Schwellenwert.\\

Als optionaler Output können über die Konfigurationsdatei und die Snakemake-Regel \lstinline|rule noderad| auch die detaillierten Graphinformationen sowie eine Visualisierung des Graphen ausgegeben werden. Die Graphinformationen wie Knoten, Kanten und ihre Eigenschaften können dabei im GraphMl-, DOT-, GML- oder im binären gt-Format gespeichert werden. Die graphische Darstellung wird als pdf-Datei ausgegeben, dabei entspricht die Kantenfärbung der berechneten Likelihood zwischen den Reads. \\

\subsection{Bestimmung der Zusammenhangskomponenten} \label{subsec:comp}

Die Bestimmung und Indexierung der Zusammenhangskomponenten erfolgt durch graph-tool ~\cite{docs_graph_tool}. Die Indexnummer jeder Zusammenhangskomponente wird den in ihr enthaltenen Knoten als Knoteneigenschaft hinzugefügt. Zusammenhangskomponenten mit mehr als einem Knoten  werden als neuer eigenständiger Graph initialisiert und in einer Liste abgelegt. Hierfür wird aus dem Graphen für jede Komponente eine gefilterte Sicht erzeugt, die als neues Graph-Object gespeichert wird. 

Da alle weiteren Schritte des Algorithmus jeweils auf den einzelnen Komponenten durchgeführt werden, kann durch die Verwendung einer Liste von Graphen im Folgenden eine einfachen Iteration über die Komponenten ausgeführt werden, ohne dass der Filtervorgang über alle Knoten jeder Komponente wiederholt werden muss. Zudem ermöglicht diese Datenstruktur eine effizientere Traversierung und Suche innerhalb der Zusammenhangskomponente, ohne dass für jede Komponente der gesamte Graph betrachtet werden muss. Der ursprüngliche Graph wird anschließend entfernt, um Arbeitsspeicher freizugeben. \\

In der Log-Datei wird die Anzahl der Knoten aller Zusammenhangskomponenten als Histogramm festgehalten. Ebenso wird dort für alle Komponenten mit mehr als einem Element die Anzahl ihrer Knoten, Kanten und Eigenschaften aufgelistet.\\

Über die Konfigurationsdatei und die Snakemake-Regel \lstinline|rule noderad| können optional auch für die Zusammenhangskomponenten jeweils Visualisierungen und detaillierte Graphinformationen in den oben genannten Formaten (Kap. \ref{subsec:edges}) ausgegeben werden. Zudem kann optional auch der gesamte Graph mit den Komponentenindizes als Knoteneigenschaften gespeichert werden. In der visuellen Darstellung werden seine Knoten entsprechend der zugehörigen Zusammenhangskomponente eingefärbt, seine Kantenfärbung richtet sich weiterhin nach der aus \eqref{eqn:3-3} resultierenden Likelihood.

\section{Bestimmung der maximalen Likelihood der Allele} \label{sec:max_lh_allele}
\subsection{Bestimmung der Allele und ihrer möglichen Häufigkeitsverteilung} \label{subsec:cand_allele}

Die einzelnen Zusammenhangskomponenten repräsentieren einen oder mehrere Loci. Alle weiteren Berechnungen aus diesem und den folgenden Kapiteln ~\ref{sec:max_lh_loci} und ~\ref{sec:vcf} werden für jede Komponente einzeln durchgeführt, dabei sollen die wahrscheinlichsten Allelsequenzen und die dazugehörigen Loci zu identifiziert werden.\\

Zunächst werden mit der Funktion \lstinline|get_candidate_alleles()| aus dem Modul \linebreak \lstinline|likelihood_operations.py| die Allelsequenzen ermittelt, die in der Zusammenhangskomponenten vorkommen. Für deterministischere Ergebnisse werden diese lexikographisch sortiert in Form einer Liste durch die Funktion zurückgegeben. Übersteigt die Größe des Clusters, d.h. die Knotenanzahl einer Zusammenhangskomponenten einen in der Konfigurationsdatei als \lstinline|treshold-cluster-size| festgelegten Wert, so wird von den im Cluster vorkommenden Sequenzen zunächst die absolute Häufigkeit bestimmt. Es werden dann nur diejenigen Sequenzen lexikographisch sortiert zurückgegeben, die einen weiteren, ebenfalls in der Konfigurationsdatei festgelegten Schwellenwert, \lstinline|treshold-seq-noise|, überschreiten. Dies dient dazu, bei großen Clustern Komplexität und Rechenaufwand zu reduzieren, da davon auszugehen ist, dass innerhalb eines großen Clusters echte Varianten einer Sequenz mehrfach vorkommen, wohingegen Artefakte und Sequenzierfehler eher vereinzelt auftreten. Dieses sog. Rauschen kann durch Anpassung der Schwellwerte in der Konfigurationsdatei herausgefiltert werden. Da bei kleineren Clustern auch einmalig registrierte Varianten einer Sequenz von Bedeutung sein können, soll der Filtervorgang erst ab einer festlegbaren Clustergröße durchgeführt werden. \\

Aus der so erzeugten Liste lexikographisch sortierter Kandidatenallele $ A_{observed} $ der Länge $ n_{observed} $ sollen nun die möglichen Häufigkeitsverteilungen in Abhängigkeit von der Ploidie bestimmt werden. Hierzu muss zunächst die aufgrund der Ploidie tatsächlich zu erwartende Anzahl von Allelen $n_{alleles}$ bestimmt werden. Die geschieht durch die Funktion \lstinline|get_max_parsimony_n_alleles()| in \lstinline|likelihood_operations.py|. Dadurch werden unnötige bzw. nicht mögliche Allelkombinationen eingespart und in der weiteren Berechnung nicht berücksichtigt. Ist die Ploidie $ \phi $ höher als die Anzahl der beobachteten Allele, so muss es mindestens genauso viele und bei Homozygotie auch mehrfach vorkommende Allele geben, damit die Ploidie erfüllt werden kann. Es muss also gelten $ n_{alleles} = \phi $. Wurden dagegen mehr Allele beobachtet als aufgrund der Ploidie möglich sind und die Ploidie ist Teiler von $n_{observed}$, so können alle beobachteten Allele auch tatsächlich vorkommen, da die Zusammenhangskomponente auch mehrere Loci enthalten kann. Es gilt dann also $ n_{alleles} = n_{observed} $. Ist dagegen die Anzahl der beobachteten Allele höher als die Ploidie, aber nicht ganzzahlig durch die Ploidie teilbar, so muss die Anzahl der Allele entsprechend erhöht werden. Das heißt, es müssen tatsächlich so viele Allele vorkommen, dass eine korrekte Ploidie erreicht wird, dass also die Ploidie eine Teiler von $ n_{alleles} $ wird. Die Anzahl der Allele muss also um die Ploidie abzüglich des Restes aus der Restdivision erhöht werden: $ n_{alleles} = n_{observed} + \phi - (n_{observed} \mod \phi)$. \\

Mit Hilfe der tatsächlich zu erwarteten Anzahl von Allelen $ n_{alleles} $ können nun alle Kombinationen möglicher Häufigkeitsverteilungen der Allele bestimmt werden (Funktion \linebreak \lstinline|get_candidate_vafs()| im Modul \lstinline|likelihood_operations.py|). Hierfür werden zunächst alle möglichen Allelkombinationen ermittelt. Dies erfolgt nach dem Urnenmodell unter Auswahl von $ n_{alleles} $ Elementen mit Zurücklegen aus insgesamt $ n_{observed} $ verschiedenen Elementen und ohne Berücksichtigung der Reihenfolge. Dadurch berechnet sich die Anzahl möglicher Kombination nach \eqref{eqn:3-12} aus dem  Binomialkoeffizienten der $k$-ten Ordnung aus $ n $ Elementen mit Zurücklegen ~\cite{tb_stat,bronst}.
\begin{equation} \label{eqn:3-12}
\tag{3-12}
\binom{n + k - 1}{k} = \frac{(n+k-1)!}{(n-1)!\, \cdotp k!} = \frac{(n_{alleles}+n_{observed}-1)!}{(n_{alleles}-1)!\, \cdotp n_{observed}!} 
\end{equation}

Diese Allelkombinationen werden mit Hilfe der Funktion  \lstinline|combinations_with_replacement| aus der Python-Library \lstinline|itertools| erzeugt \cite{itertools}.\\
\\
\definecolor{light-gray}{gray}{0.93}
\fcolorbox{black}{light-gray}{
	\parbox{\textwidth}{
		\vspace{0.5cm}
		\textbf{Beispiele möglicher Allelkombinationen:} \\	
		\\	
		$ ploidy = 2, n_{observed} = 2, n_{alleles} = 2$: \\
		{[(0, 0), (0, 1), (1, 1)]}\\
		\\
		$ ploidy = 2, n_{observed} = 3, n_{alleles} = 4$: \\
		{[(0, 0, 0, 0), (0, 0, 0, 1), (0, 0, 0, 2), (0, 0, 1, 1), (0, 0, 1, 2), (0, 0, 2, 2), (0, 1, 1, 1), (0, 1, 1, 2), (0, 1, 2, 2), (0, 2, 2, 2), (1, 1, 1, 1), (1, 1, 1, 2), (1, 1, 2, 2), (1, 2, 2, 2), (2, 2, 2, 2)]}
		\\
	}}\\


Für jedes Allel $i$ können im Anschluss aus seinen absoluten Häufigkeiten $H_{a_{i}}$ innerhalb jeder Allelkombination die relativen Häufigkeiten $h_{a_{i}}$ nach \eqref{eqn:3-13} bestimmt werden. 
\begin{equation} \label{eqn:3-13}
\tag{3-13}
h_{a_{i}} = \frac{H_{a_{i}}} {n_{alleles}}
\end{equation}
\\		
\fcolorbox{black}{light-gray}{
	\parbox{\textwidth}{
		\vspace{0.5cm}
		\textbf{Beispiele möglicher Häufigkeitsverteilungen der Allele:} \\	
		\\	
		$ ploidy = 2, n_{observed} = 2, n_{alleles} = 2$: \\
		{[1.0, 0.0]}, {[0.5, 0.5]}, {[0.0, 1.0]} \\
		\\
		$ ploidy = 2, n_{observed} = 3, n_{alleles} = 4$: \\
		{[1.0, 0.0, 0.0]}, {[0.75, 0.25, 0.0]}, {[0.75, 0.0, 0.25]}, {[0.5, 0.5, 0.0]}, {[0.5, 0.25, 0.25]}, {[0.5, 0.0, 0.5]}, {[0.25, 0.75, 0.0]}, {[0.25, 0.5, 0.25]}, {[0.25, 0.25, 0.5]}, {[0.25, 0.0, 0.75]}, {[0.0, 1.0, 0.0]}, {[0.0, 0.75, 0.25]}, {[0.0, 0.5, 0.5]}, {[0.0, 0.25, 0.75]}, {[0.0, 0.0, 1.0]}
		\\
}}
\linebreak 

\subsection{Berechnung der Likelihoods der Allele anhand der möglichen Häufigkeitsverteilungen} \label{subsec:lh_allele}

Nun soll nach \eqref{eqn:2-3} die Wahrscheinlichkeit ermittelt werden, einen Read anhand der gegebenen Allelfraktion zu beobachten (vgl. Kap. \ref{subsec:sol_allele_lh}). Dies geschieht in der Funktion \lstinline|get_allele_likelihood_read()| aus dem Modul \lstinline|likelihood_operations.py|). Für jeden Read wird dabei zunächst die Wahrscheinlichkeit bestimmt, dass der Read durch Sequenzierfehler aus einem der Kandidatenallele entstanden ist. Hierzu wird für jeden Read $ r_{i} $ jeweils eine ausgehende Kante zu jedem Kandidatenallel $ a_{j} $ gesucht und die nach \eqref{eqn:3-2} berechnete Likelihood aus den Eigenschaften dieser Kante zurückgegeben.\\

Geordnet nach dem Laufzeitaufwand werden für das Auffinden einer solchen Kante verschiedene Möglichkeiten geprüft:
\begin{enumerate}
	\item \label{-1-} Der Read besitzt eine ausgehende Kante zu einem Knoten, der mit der Sequenz des Kandidatenallels übereinstimmt. Hierfür werden alle ausgehenden Nachbarn des Knotens hinsichtlich ihrer in den Knoteneigenschaften abgelegten Sequenz geprüft. Wurde eine solche Kante gefunden, so wird die Likelihood der Kante direkt zurückgegeben. Andernfalls erfolgt die Suche wie unten beschrieben. 
	\item \label{-2-} Existiert keine Kante vom Knoten des Reads zu einem Knoten mit der Kandidatenallelsequenz, so wird nach einer entgegengesetzt gerichteten Kante gesucht. Dazu werden alle eingehenden Nachbarn auf die Sequenz des Kandidatenalles überprüft. Falls es eine solche Kante gibt, dann kann die gesuchte Likelihood der Gegenrichtung, über die Funktion \lstinline|get_alignment_likelihood()| mit dem Argument \lstinline|reverese = True| bestimmt werden.
	\item \label{-3-} Falls auch in Gegenrichtung keine passende Kante gefunden wurde, so wird in der gesamten Zusammenhangskomponente nach einer Kante gesucht, die zwei Knoten verbindet, deren Sequenzen denen von $ r_{i} $ und $ a_{j} $ entsprechen. Existiert eine solche Kante, so werden ihre CIGAR-Tupel verwendet, um für $ r_{i} $ die Likelihood zu bestimmen. Da die CIGAR-Tupel allein über die Sequenz definiert werden, gelten die so ermittelten CIGAR-Tupel aufgrund identischer Sequenzen der begrenzenden Knoten auch für eine nicht existente Kante zwischen $ r_{i} $ und $ a_{j} $. Die CIGAR-Tupel können also verwendet werden, um zusammen mit den nach  \eqref{eqn:3-1} berechneten p-Werten des Phred Quality Scores  von $ r_{i} $ die Likelihood mittels \lstinline|get_alignment_likelihood()| zu bestimmen.
	\item \label{-4-} Existiert auch im gesamten Graphen keine vergleichbare Kante, die $ r_{i} $ und $ a_{i} $ mit einander verbindet, so wird der Wert Null zurückgegeben.
\end{enumerate}

Zur Veranschaulichung ist der beschriebene Algorithmus zudem in Pseudocode dargestellt \ref{alg:r-a-lh}. Dabei sind $ C_{k} \in \{C_{1}, \dots ,C_{p}\} $ der Subgraph einer Zusammenhangskomponente, $ r_{i} \in C_{k} $ der Knoten des Reads und $ a_{j} \in A $ ein Kandidatenallel mit der Sequenz $ s_{a_{j}} $, wobei $ k,i,j \in \mathds{N} $. Die Verwendung fester Knoten- oder Kanteneigenschaften ist durch eckige Klammern gekennzeichnet, so ruft beispielsweise $r_{i}[p\_values]$ die $ p $-Werte des Phred Quality Scores von $r_{i}$ auf.\\

\renewcommand{\algorithmiccomment}[1]{\hfill$\triangleright$\textit{#1}}
\begin{algorithm}[H]
	\caption{Bestimmung der Likelihood zwischen einem Read und einem Kandidatenallel} \label{alg:r-a-lh}
	\begin{algorithmic}[1]	
		\Function{get\_allele\_likelihood\_read}{$ C_{k} $, $ r_{i} $, $ s_{a_{j}} $}
		\State $ out\_neighbors \gets get\_out\_neighbors(r_{i})$
		\For {$out\_neighbor \in out\_neighbors $}
		\If{$ s_{a_{j}} = out\_neighbor[sequence] $}
		\State \Return $ edge(r_{i}, out\_neighbor)[likelihood]$
		\EndIf
		\EndFor
		\State $ m\_{rates} \gets C_{k}[m_{sub},m_{ins},m_{del}] $
		\State $ qual \gets r_{i}[p\_values] $
		\State $ in\_neighbors \gets get\_in\_neighbors(r_{i})$		
		\For {$in\_neighbor \in in\_neighbors $}
		\If{$ s_{a_{j}} =  in\_neighbor[sequence] $}		
		\State $ cig \gets edge(in\_neighbor, r_{i})[CIGAR-Tuples]$		
		\State $ rev \gets True $
		\State \Return $get\_alignment\_likelihood(m\_{rates},\, cig,\, qual,\, rev)$   \algorithmiccomment{Alg. \ref{alg:lh_read}}
		\EndIf		
		\EndFor
		\State $ cigar \gets get\_cigar\_tuples(C_{k},\, r_{i}[sequence],\, s_{a_{j}}) $ \algorithmiccomment{Alg. \ref{alg:cig}}		
		\If{cigar exists}
		\State $ cig \gets cigar[0]$
		\State $ rev \gets cigar[1]$
		\State \Return $get\_alignment\_likelihood(m\_{rates},\, cig,\, qual,\, rev)$   \algorithmiccomment{Alg. \ref{alg:lh_read}}
		\EndIf
		\State \Return 0
		\EndFunction
	\end{algorithmic}
\end{algorithm}

Wie bereits erwähnt, kann die Bestimmung der Wahrscheinlichkeit, dass ein Read durch Sequenzierfehler aus einem bestimmten Allel entstanden ist, erfordern, die CIGAR-Tupel des Sequenzvergleichs zu identifizieren. Auch im Folgenden (Kap. \ref{sec:max_lh_loci}) kann die Zuordnung der Kandidatenallele zu bestimmten Loci im Hinblick auf die Heterozygotiewahrscheinlichkeiten, die Ermittlung der CIGAR-Tupel zwischen den Kandidatenallelen bedingen. Dies wird in NodeRAD durch die Methode \lstinline|get_cigar_tuples()| bewerkstelligt, die sich ebenfalls im Modul \lstinline|likelihood_operations.py| befindet. Sie benötigt als Eingabeparameter eine Graphen oder Subgraphen sowie zwei Sequenzen: die Query-Sequenz $ s_{source} $ und die Referenz-Sequenz $ s_{target} $. Der Graph wird nach einer Kante durchsucht, die Knoten miteinander verbindet, welche die beiden gegebenen Sequenzen besitzen. Dabei werden nicht nur Kanten berücksichtigt, die von der Query- zur Referenzsequenz verlaufen, sondern auch entgegen gerichtete Kanten, die von der Referenz- zur Querysequenz verlaufen. Zur Markierung der Kantenrichtung gibt die Funktion neben dem CIGAR-Tupel auch einen booleschen Wert zurück. Dieser wird in den Funktionen \lstinline|get_alignment_likelihood()| (Kap. \ref{subsec:edges}) und \lstinline|get_heterozygosity()| (Kap. \ref{subsec:lh_loci}) für die Likelihoodberechnung benötigt. Im Falle einer entgegen gerichteten Kante, also bei \lstinline|reverse=True|, werden Deletionen als Insertionen gewertet und umgekehrt. Bei erfolgreicher Sucher gibt die Funktion neben dem booleschen Wert auch die gefundenen CIGAR-Tupel zurück, ansonsten wird der Wert $ None $ zurückgegeben.\\

Um eine passende Kante im Graphen zu finden, erstellt \lstinline|get_cigar_tuples()| zunächst eine Liste aller Knoten $ R_{source} = (v_{1}, \dots, v_{i})$, welche die Query-Sequenz $ s_{source} $ besitzen sowie eine Liste aller Knoten $ R_{target} = (w_{1}, \dots, w_{j}) $, die die Referenz-Sequenz $ s_{target} $ tragen. Hierfür werden für jede Sequenz alle Knoten des Graphen mit \lstinline|find_vertex()| aus graph-tool durchsucht. Nun wird nach einer Kante gesucht, die einen der Knoten aus $ R_{source} $ mit einem der Knoten aus $ R_{target} $ verbindet. Es erfolgt also ein Kantenaufruf für jeden Knoten aus $ R_{source} $ in Kombination mit jedem  Knoten aus $ R_{target} $. Der Kantenaufruf erfolgt sowohl für die angegebene Kantenrichtung und falls erfolglos auch für die entgegengesetzte Richtung. Existiert eine solche Kante, dann werden ihre CIGAR-Tupel sowie der boolsche Wert entsprechend der Kantenrichtung zurückgegeben. \\

Zur Veranschaulichung ist der Algorithmus in \ref{alg:cig} zudem als Pseudocode dargestellt. Dabei sind $ C_{k} \in \{C_{1}, \dots ,C_{p}\} $ der Subgraph einer Zusammenhangskomponente, $ s_{query} $ die Query-Sequenz und $ s_{ref} $ die Referenz-Sequenz. Die Verwendung fester Knoten- oder Kanteneigenschaften ist durch eckige Klammern gekennzeichnet, so ruft beispielsweise $v_{i}[sequence]$ die Sequenz des Knotens $v_{i}$ aus den Knoteneigenschaften ab. \\

\begin{algorithm}[H]
	\caption{CIGAR-Tupel bestimmen}  \label{alg:cig}
	\begin{algorithmic}[1]	
		\Function{get\_cigar\_tuples}{$ C_{k} $, $ s_{query} $, $ s_{ref} $}
		\State $ R_{source} \gets \{v_{i} \in C_{k} \wedge i,k \in \mathds{N} \; |\; v_{i}[sequence]= s_{query} \}$
		\State $ R_{target} \gets \{w_{j} \in C_{k} \wedge j,k \in \mathds{N} \; |\; w_{j}[sequence]= s_{ref} \}$
		\For{$ v_{i} \in R_{source} $}
		    \For{$ w_{j} \in R_{target} $}
		        \If{$ edge(v_{i}, w_{j})$ exists}
		            \State \Return $ (\;edge(v_{i}, w_{j})[CIGAR-Tuples],\, False\;) $		    
		        \EndIf
		        \If{$ edge(w_{j}, v_{i})$ exists}
		            \State \Return $ (\;edge(w_{j}, v_{i})[CIGAR-Tuples],\, True\;) $		
		        \EndIf
		    \EndFor
		\EndFor
		\State \Return $ None $
		\EndFunction
	\end{algorithmic}
\end{algorithm}

In  \lstinline|get_allele_likelihood_read()| wurden also die Wahrscheinlichkeiten jedes Reads bestimmt, dass er von den verschiedenen Allelen der Zusammenhangskomponente stammt (siehe oben). Die Wahrscheinlichkeiten im Bezug auf jedes Allel sollen nun in die zuvor bestimmten Kombinationen der Häufigkeitsverteilungen der Allele (siehe Kap. \ref{subsec:cand_allele}) nach Formel \eqref{eqn:2-xxx2} einbezogen werden. In der Funktion \lstinline|calc_vafs_likelihood_read()| aus dem Modul \lstinline|likelihood_operations.py| wird also für jeden Read die Wahrscheinlichkeit berechnet, diesen unter der gegebenen Allelfraktion zu beobachten. Dazu wird über die relativen Häufigkeiten jedes Allels einer gegebenen VAFs-Kombination iteriert und ihr Produkt mit der Readlikelihood aufsummiert. \\

Aus dem Produkt über alle Reads wird anschließend für eine gegebene Allelfraktion nach Formel \eqref{eqn:2-xxx3} in \lstinline|calc_vafs_likelihood()| die Gesamtwahrscheinlichkeit errechnet. \\

Auf diese Weise wird in \lstinline|noderad_main.py| für jede Allelfraktion die Gesamtwahrscheinlichkeit über alle Reads und alle relativen Häufigkeiten der Allele berechnet und in einer Liste abgelegt. \\

Aus dieser Liste wird die Allele-Fraction mit maximaler Likelihood als wahrscheinlichste Lösung gewählt und in Kap. \ref{sec:max_lh_loci} verwendet, um die wahrscheinlichsten Loci zu ermitteln, die diese Häufigkeitsverteilung der Allele erklären können.  \\

Als zusätzliche Statistiken werden in den Log-Dateien für jede Komponente die Anzahl der Allele, die Ploidie sowie die Häufigkeitsverteilung Allelfraktion mit der maximaler Likelihood und den dazugehörigen Allelsequenzen festgehalten.

\section{Bestimmung der maximalen Likelihood der Loci} \label{sec:max_lh_loci}

In Abhängigkeit von der Anzahl der Kandidatenallele und der Ploidie können die Zusammenhangskomponenten auch mehr als nur einen Locus beinhalten. Daher soll die in Kap. \ref{sec:max_lh_allele} identifizierte Allelfraktion mit maximaler Likelihood nun passenden Loci-Kombinationen zugeordnet werden. Unter Einbeziehung der Heterozygotiewahrscheinlichkeiten soll daraus die wahrscheinlichste Loci-Zuordnung ermittelt werden.

\subsection{Bestimmung der möglichen Loci} \label{subsec:comb_loci}

Für die Zuordnung der Loci müssen zunächst die verschiedenen Kombinationen möglicher Allelverteilungen auf die Loci durch \lstinline|get_candidate_loci()| im Modul \linebreak \lstinline|likelihood_operations.py| erzeugt werden. Analog zu \lstinline|get_candidate_vafs()| (Kap. \ref{subsec:cand_allele}) werden aus der beobachteten $n_{observed} $ und der tatsächlich zu erwartenden Anzahl an Allelen $n_{alleles}$ alle Allelkombinationen mit Wiederholung und ohne Berücksichtigung der Reihenfolge  (vgl. Formel \eqref{eqn:3-12}) durch die Funktion \lstinline|combinations_with_replacement()| aus der Python-Library \lstinline|itertools| gebildet. Allerdings sollen nun die Allelkombinationen zu Tupeln entsprechend ihrer Ploidie zusammengefasst werden. Jedes Tupel repräsentiert einen möglichen Locus in der Kombination. Die Allelkombination durch \linebreak \lstinline|combinations_with_replacement()| sind lexikographisch sortiert. Für jede der Kombinationen muss nun aber die Reihenfolge so variiert werden, dass die Allele, falls mehrere Loci möglich sind, in verschiedenen Kombinationen den Loci zugeordnet werden. Hierfür wird aus \lstinline|itertools| die Funktion \lstinline|permutations| verwendet. Diese bildet aus $ n $ Elementen alle Kombinationen der Länge $ k $ mit Beachtung der Reihenfolge und ohne Wiederholungen. Auch wenn sich die Allele in den Allelkombinationen durchaus wiederholen können, so werden sie doch in \lstinline|permutations| als einzigartiges Element behandelt unabhängig von jeweiligen Wert. Folglich entstehen durch die Bildung der Permutationen auch einige Duplikate. Die Anzahl der entstehenden Elemente wird für \lstinline|permutations| mit  $ \frac{n!}{(n-k)!} $ angegeben \cite{itertools}. Die Permuationen werden über jede Allelkombination in ihrer vollständigen Länge gebildet, so dass gilt $ n = k $, folglich werden insgesamt $ n! $ Elemente erzeugt \eqref{eqn:3-28}. 
\begin{equation} \label{eqn:3-28}
\tag{3-28}
\frac{n!}{(n-k)!}=\frac{n!}{(n-n)!}=\frac{n!}{0!}=\frac{n!}{1}=n!
\end{equation}
Mit Hilfe der \lstinline|itertools|-Funktion \lstinline|grouper()| werden die Permutationen entsprechend der Ploidie in Tupel der Länge $\phi$ unterteilt. Es resultieren Tupel von $ \phi $-Tupeln, wobei die Anzahl der $phi$-Tupel der Anzahl möglicher Loci entspricht. Mehrere Loci in einer Zusammenhangskomponenten sind nach \lstinline|get_max_parsimony_n_alleles()| genau dann möglich, wenn gilt $n_{observed} > \phi $.\\

Nach der Gruppierung werden die übergeordneten Tupel sowie die darin enthaltenen $\phi$-Tupel jeweils sortiert, um anschließend durch die Python-Funktion \lstinline|set()| die oben erwähnten Duplikate zu entfernen.

	\fcolorbox{black}{light-gray}{
	\parbox{\textwidth}{
		\vspace{0.5cm}
		\textbf{Beispiele möglicher Loci-Kombinationen:} \\	
				
		$ ploidy = 2, n_{observed} = 3, n_{alleles} = 4 $: \\
		
		\textbf{Kombinationen der Allele:}\\
		{[(0, 0, 0, 0), (0, 0, 0, 1), (0, 0, 0, 2), (0, 0, 1, 1), (0, 0, 1, 2), (0, 0, 2, 2), (0, 1, 1, 1), (0, 1, 1, 2), (0, 1, 2, 2), (0, 2, 2, 2), (1, 1, 1, 1), (1, 1, 1, 2), (1, 1, 2, 2), (1, 2, 2, 2), (2, 2, 2, 2)]}\\
		
		\textbf{Permutationen:} \\
		{\small (Zur anschaulicheren Darstellung wurden Duplikate entfernt, das geschieht eigentlich erst nach der Erzeugung und Sortierung der Tupel)\\}
		(1, 2, 1, 1), (2, 1, 0, 0), (2, 1, 1, 1), (0, 1, 2, 1), (0, 1, 1, 2), (0, 1, 0, 0), (2, 2, 1, 0), (0, 2, 2, 1), (2, 2, 0, 1), (1, 0, 2, 2), (0, 2, 0, 1), (2, 0, 0, 1), (1, 0, 1, 0), (0, 2, 1, 2), (0, 0, 2, 0), (2, 2, 2, 1), (1, 1, 0, 1), (2, 0, 1, 1), (2, 0, 2, 0), (0, 0, 2, 2), (1, 1, 2, 0), (1, 2, 1, 0), (2, 0, 2, 2), (2, 1, 1, 0), (2, 1, 0, 2), (1, 2, 0, 1), (0, 1, 2, 0), (1, 2, 1, 2), (1, 2, 2, 1), (0, 1, 1, 1), (1, 1, 1, 0), (0, 0, 0, 0), (2, 1, 1, 2), (2, 1, 2, 1), (1, 0, 0, 1), (0, 1, 0, 2), (2, 2, 1, 2), (0, 2, 2, 0), (1, 0, 2, 1), (2, 0, 0, 0), (0, 2, 1, 1), (1, 1, 1, 2), (0, 0, 0, 2), (0, 0, 1, 1), (1, 0, 1, 2), (2, 0, 0, 2), (0, 0, 2, 1), (1, 1, 2, 2), (2, 1, 0, 1), (1, 2, 0, 0), (0, 1, 2, 2), (1, 2, 2, 0), (0, 1, 1, 0), (2, 2, 0, 0), (0, 2, 0, 0), (2, 1, 2, 0), (1, 0, 0, 0), (1, 2, 0, 2), (0, 1, 0, 1), (2, 2, 1, 1), (2, 2, 2, 0), (1, 1, 0, 0), (1, 0, 2, 0), (0, 2, 2, 2), (1, 2, 2, 2), (2, 2, 0, 2), (0, 2, 1, 0), (0, 2, 0, 2), (2, 0, 1, 0), (0, 0, 0, 1), (1, 1, 1, 1), (0, 0, 1, 0), (2, 1, 2, 2), (1, 0, 0, 2), (1, 0, 1, 1), (2, 2, 2, 2), (1, 1, 0, 2), (2, 0, 1, 2), (2, 0, 2, 1), (0, 0, 1, 2), (1, 1, 2, 1)\\	    
		
		\textbf{Loci}:\\
		((0, 0), (0, 2)), ((1, 1), (1, 1)), ((0, 2), (0, 2)), ((1, 1), (2, 2)), ((0, 1), (0, 2)), ((1, 1), (1, 2)), ((1, 2), (2, 2)), ((1, 2), (1, 2)), ((0, 0), (1, 1)), ((0, 0), (2, 2)), ((0, 2), (1, 1)), ((0, 1), (1, 1)), ((0, 0), (0, 0)), ((0, 2), (2, 2)), ((0, 0), (1, 2)), ((0, 1), (2, 2)), ((2, 2), (2, 2)), ((0, 0), (0, 1)), ((0, 2), (1, 2)), ((0, 1), (1, 2)), ((0, 1), (0, 1)) \\
}}\\

\subsection{Zuordnung der Allele-Fractions mit maximaler Likelihood zur wahrscheinlichsten Loci-Kombination} \label{subsec:lh_loci}

In diesem Schritt sollen schließlich für die Allelfraktion mit maximaler Likelihood (Kap.\ref{subsec:lh_allele}) die ermittelten Kandidatenloci (Kap. \ref{subsec:comb_loci}) unter Berücksichtigung der Heterozygotie analysiert werden und die wahrscheinlichste Kombination der Loci ermittelt werden.\\

Für jede mögliche Kombination der Loci aus \lstinline|get_candidate_loci()| wird aus der Funktion \lstinline|calc_loci_likelihoods()| heraus zunächst geprüft, ob sich die Loci dieser Kombination der zuvor bestimmten Allelfraktion mit maximaler Likelihood zuordnen lassen. Dies geschieht mit Hilfe der Indikatorfunktion \lstinline|indicator_constrait()| nach Formel \eqref{eqn:2-xxx4}. Ist die Bedingung der Indikatorfunktion erfüllt, so wird die Likelihood der Loci-Kombination aus den Heterozygotiewahrscheinlichkeiten der darin enthaltenen Allele berechnet und zurückgegeben. Ist die Bedingung der Indikatorfunktion nicht erfüllt, so wird eine Likelihood von 0 zurückgegeben.\\

Für die Likelihoodberechnung bei erfüllter Indikatorbedingung müssen einer Loci-Kombination zunächst die passenden Allelsequenzen zugeordnet werden. Die Allelnummer in einer Loci-Kombination entspricht dabei dem Index der dazugehörigen Allelsequenz in der Liste der Kandidatenallele aus \lstinline|get_candidate_alleles()| (siehe Kap. \ref{subsec:cand_allele}). Zu jeder Loci-Kombination wird also eine Liste $ L_{alleles} $ generiert, welche die Kandidatenallelsequenzen der einzelnen Loci als Sublisten beinhaltet.\\

Analog zur Likelihoodberechnung beim paarweisen Vergleich der Reads unter Berücksichtigung der Sequenzierfehlerwahrscheinlichkeit \ref{subsec:edges} erfolgt auch die Berechnung der Likelihood für eine bestimmte Loci-Kombination durch den paarweisen Vergleich der Allele im Sinne eines pair Hidden Markov Models $ pairHMM_{\eta}(a_{l_{j,1}}, a_{l_{j,2}}) $ (vgl. Kap. ~\ref{subsec:sol_phmm}). Allerdings werden nun die Heterozygotiewahrscheinlichkeiten $\eta_{sub}$, $\eta_{ins}$ und $\eta_{del}$ der Grapheigenschaften zur Ermittlung der Likelihood nach Formel \eqref{eqn:2-xxx5} genutzt.  \\

Hierzu werden zunächst in \lstinline|get_allele_likelihood_alleles()| (Algorithmus \ref{alg:lh_loci}) auf jeweils einer der Listen $L_{alleles}$ für jeden Locus die darin enthaltenen Allelsequenzen paarweise mit einander kombiniert. Dies wird durch die Funktion \lstinline|combinations()| aus \lstinline|itertools| realisiert, welche Kombinationen ohne Wiederholung und ohne Beachtung der Reihenfolge erzeugt. Für jedes Allelpaar wird dann der CIGAR-String durch die in Kap. ~\ref{subsec:lh_allele} bereits beschriebene Funktion \lstinline|get_cigar_tuples()| ermittelt und für die Likelihoodberechnung in \lstinline|get_heterozygosity()| verwendet (siehe Algorithmus \ref{alg:lh_het}). Die Gesamtwahrscheinlichkeit errechnet sich aus dem Produkt der Likelihood aller Allelpaare nach \ref{eqn:2-xxx6}.

\begin{algorithm}[H]
	\caption{Bestimmung der Likelihood der Allele innerhalb einer Loci-Kombination}  \label{alg:lh_loci}
	\begin{algorithmic}[1]	
		\Function{get\_allele\_likelihood\_alleles}{$ C_{k} $, $ L_{alleles} $}
		\State $ likelihood \gets 1.0 $
		\For{$ loc \in L_{alleles} $}
		    \State $ pairs \gets \binom{loc}{2}$
		    \For{$ (a_{i}, a_{j}) \in pairs $}
		    \State $cigar \gets get\_cigar\_tuples(C_{k}, a_{i}, a_{j})$  \algorithmiccomment{Alg. \ref{alg:cig}}
			    \If{$cigar$ exists}
			        \State $cig \gets cigar[0]$
			        \State $rev \gets cigar[1]$
			        \State $ \eta_{rates} \gets C_{k}[\eta_{sub}, \eta_{ins}, \eta_{del}]$
			        \State $pr \gets get\_heterozygosity(C_{k}[\eta_{sub}], C_{k}[\eta_{ins}], C_{k}[\eta_{del}], cig, rev)$  \algorithmiccomment{Alg. \ref{alg:lh_het}}
			    	\State $ likelihood \gets likelihood + log(pr)$    
			    \EndIf
			\EndFor
		\EndFor
		\State \Return $ e^{\;likelihood} $
		\EndFunction
	\end{algorithmic}
\end{algorithm}

Wie bereits erwähnt, erfolgt in \lstinline|get_heterozygosity()| die Likelihoodberechnung der paarweisen Kombinationen der Allelsequenzen eines Locus aus den CIGAR-Tupeln und den Heterozygotiewahrscheinlichkeiten. Analog zum Algorithmus ~\ref{alg:lh_read} wird auch hier die Likelihood über aller Matches und Mismatches berechnet.\\

\begin{algorithm}[H]
	\caption{Bestimmung der Likelihood zwischen zwei Allelen hinsichtlich der Heterozygotiewahrscheinlichkeiten}  \label{alg:lh_het}
	\begin{algorithmic}[1]	
		\Function{get\_heterozygosity}{$\eta_{sub}$, $\eta_{ins}$, $\eta_{del}$, $ CIGAR-Tuples $, $ reverse $}
		\State $likelihood \gets 1.0$
		\If{reverse}
		\State swap values of $\eta_{ins}$ and $\eta_{del}$
		\EndIf
		\State $\eta_{all} \gets \eta_{sub}+\eta_{ins}+\eta_{del}$
		\ForAll{$(operation, length) \in CIGAR-Tuples$}
		\If{$operation \in match$}
		\State $ likelihood \gets likelihood \, \cdotp (1 - \eta_{all})^{\;length} $
		\EndIf		    
		\If{$operation \in mismatch$}
		\State $ \eta_{factor} \gets 1.0$			    	
		\If{$operation \in substitution$}
		\State $\eta_{factor} \gets \eta_{sub}$
		\EndIf	
		\If{$operation \in insertion$}
		\State $\eta_{factor} \gets \eta_{ins}$
		\EndIf
		\If{$operation \in deletion$}
		\State $\eta_{factor} \gets \eta_{del}$
		\EndIf
		\State $ likelihood \gets likelihood \, \cdotp (\eta_{factor})^{length}$
		\EndIf
		\EndFor
		\State \Return $ likelihood $
		\EndFunction
	\end{algorithmic}
\end{algorithm}

Dabei entspricht im Falle eines Mismatches die Likelihood der in der Konfigurationsdatei angegebenen Heterozygotiewahrscheinlichkeit $ \eta $ für die betreffende Mutationsart, also $ \eta_{sub} $ für Substitutionen , $ \eta_{ins} $ für Insertionen bzw. $ \eta_{del} $ Deletionen. Sei $i$ der Index der betreffenden Base innerhalb der betrachteten Allelsequenz und $ \eta_{rate} \in \{\,\eta_{sub},\, \eta_{ins},\, \eta_{del}\,\}$, dann berechnet sich bei einem Mismatch die Likelihood $L_{i}$ der Base nach Formel \eqref{eqn:3-30}.
\begin{equation} \label{eqn:3-30}
\tag{3-30}
L_{i\,_{mismatch}} = \eta_{rate}
\end{equation}

Im Falle eines Matches senkt die Möglichkeit eines Mismatches die Likelihood der betreffenden Base entsprechend um die Summe der Heterozygotiewahrscheinlichkeiten der genannten Mutationsarten (Formel \eqref{eqn:3-31}).
\begin{equation} \label{eqn:3-31}
\tag{3-31}
L_{i\,_{match}} = 1 - (\eta_{sub} + \eta_{ins} + \eta_{del})
\end{equation}

Die Likelihood $ Pr(T=a_{l_{j,2}} \, | \, S=a_{l_{j,1}}, \eta) $ des paarweisen Vergleichs der Allele $a_{l_{j,1}}$ und $a_{l_{j,2}}$ hinsichtlich der Zuordnung zu den Loci-Kombinationen errechnet sich schließlich aus dem Produkt der Wahrscheinlichkeiten der einzelnen Basen:
\begin{equation} \label{eqn:3-32}
\tag{3-32}
Pr(T=a_{l_{j,2}} \, | \, S=a_{l_{j,1}}, \eta) = pairHMM_{\eta}(a_{l_{j,1}}, a_{l_{j,2}}) = \prod_{i=1}^{k}L_{i}
\end{equation}

In \lstinline|noderad_main.py| werden die beschriebenen Schritte zur Likelihoodberechnung für alle Permutationen ausgeführt und anschließend die Loci-Kombination mit maximaler Likelihood als wahrscheinlichste Loci-Verteilung ausgewählt.\\

In den Log-Dateien werden zusätzlich einige Statistiken zur Berechnung der wahrscheinlichsten Loci-Verteilung eingetragen. Dazu gehören die ermittelten Likelihoods aller Permutationen sowie die Loci mit maximaler Likelihood. \\

\section{Ausgabe der wahrscheinlichsten Loci als VCF-Datei} \label{sec:vcf}

Die Loci-Kombinationen mit maximaler Likelihood $Loc_{max}$ aller Zusammenhangskomponenten werden für jedes Individuum jeweils in eine Datei im Variant Call Format (VCF, \cite{danecek_2011}) geschrieben. Wie für dieses Format üblich, gibt es acht notwendige tabulatorgetrennte Spalten: CHROM, POS, ID, REF, ALT, QUAL, FILTER, INFO. Außerdem wird hier die optionale Spalte FORMAT sowie eine Spalte zur Probe verwendet. \\

Die Spalte CHROM wird hier für die Bennennung der Loci genutzt. Die Bezeichnung jedes Locus wird dabei aus dem Präfix ''LOC'' in Kombination mit einer fortlaufenden Nummer gebildet. Die Spalten ID, QUAL, FILTER und INFO werden hier nicht genutzt und mit einem Punkt als Platzhalter bei allen Loci befüllt.\\

In die Spalten REF und ALT sollen die Sequenzen der Allele aus $Loc_{max}$ lexikographisch sortiert geschrieben werden. Hierfür wird in der Funktion \lstinline|get_sorted_loci_alleles()| eine sortiere Liste der Allelsequenzen aus $Loc_{max}$ gebildet. Durch die Pythonfunktion \lstinline|set()| wird über dieser Liste die Menge der Allele extrahiert, so dass Duplikate herausgefiltert werden. Aus der so entstandenen Liste der Allelsequenzen $A_{res}$ wird der erste Eintrag in die Spalte REF eingetragen, die übrigen Einträge werden in die Spalte ALT geschrieben.\\

Da jeweils die vollständige Sequenz der Allele für den Eintag in die VCF-Datei verwendet wird, wird für jeden Locus in der Spalte POS der Wert $1$ eingetragen. \\

In der FORMAT-Spalte wird durch den Eintrag ''GT'' festgelegt, dass in der Probenspalte der Genotyp spezifiziert wird. Da die Indizes der Loci der Liste der Kandidatenallele zugeordnet sind, aber nicht alle Allele dieser Liste auch in $Loc_{max}$ vorkommen, muss ihre Indizierung nun auf die bereits erstellte Liste mit den zu $Loc_{max}$ gehörigen Allelsequenzen $A_{res}$ angepasst werden. \\

Hierfür werden in der Funktion \lstinline|get_alleles_matched_to_loci()| zunächst jedem Locus aus $Loc_{max}$ die entsprechenden Sequenzen aus der Liste der Kandidatenallele zugeordnet. Anschließend werden diese Sequenzen in \lstinline|get_gt_indices()| den Indizes der passenden Sequenz aus $A_{res}$ zugeordnet. Dadurch wird das lexikographisch erste Allel, also REF mit 0 im Genotyp indiziert, die Allele aus ALT erhalten höhere Indizes entsprechend ihrer Sortierung in $A_{res}$. Nun lässt sich der Genotyp im Bezug auf die Sequenzen in REF und ALT aus diesen Indizes und getrennt durch Slashes direkt angeben. Dies geschieht in der Funktion \lstinline|get_genotype()|, deren Ergebnis dann in die Probenspalte eingetragen wird.\\


